% !Tex root = checkedc.tex

\chapter{Generating efficient code for local variables of type \arrayview}\label{appendix-a-generating-efficient-code-for-local-variables-of-type-arrayux5fview}

This appendix gives an example of how a compiler can generate efficient
code when \arrayview\ is used in place of \arrayptr\
for local variables. The compiler can do this by applying well-known
optimizations: inlining operations for \arrayview , replacing
struct members with local variables, copy propagation, reverse copy
propagation, and eliminating identity assignments.

Here is a possible implementation of \arrayviewT (in C++). It
uses the proposed extensions to C. The italicized assert in
\texttt{Deref} represents code that is injected by the compiler to check
the bounds at run time.

\begin{verbatim}
struct array_view<T>
{
    array_ptr<T> current;
    where bounds(current) == (low, high);

    array_ptr<T> low;
    array_ptr<T> high;
      
    array_view<T>(array_ptr<T> ptrVal 
                  where bounds(ptrVal) == (lowerBound, upperBound),
                  array_ptr<T> lowerBound, array_ptr<T> upperBound)
    {
        current = ptrVal;
        low = lowerBound;
        high = upperBound
        where bounds(current) == (low, high);
    }

    T Deref(array_view<T> av) 
    {
        array_ptr<T> tmp = av.current;
        assert(tmp>= av.low && tmp < av.high);
        return *tmp;
    }

    array_view<T> Add(array_view<T> av, int delta) 
    {
        array_view<T> tmp = av;
        tmp.current = tmp.current + delta;
        return tmp;
    }
}
\end{verbatim}

Here is a version of the sum routine where the local variable buf has
type \arrayview. This means that no bounds need to be declared
for buf.

\begin{verbatim}
int sum(array_ptr<int> argbuf where bounds(argbuf) == (argbuf, end), 
        array_ptr<int> end) {
    array_view<int> buf = new array_view(argbuf, argbuf, end);
    int sum = 0;
    while (buf < end) {
        sum += *buf;   // checks that buf is in  bounds before dereferencing it        
        tmp = buf + 1; // new struct with updated pointer, same bounds
        buf = tmp;
    }
    return sum;
}
\end{verbatim}

The compiler can inline the code for the \arrayview\ struct operations
and place struct members in local variables:
\begin{verbatim}
int sum (array_ptr<int> argbuf where bounds(argbuf) == (argbuf, end),
         array_ptr<int> end) {
    // buf = new array_view(argbuf, argbuf, end);
    array_ptr<int> buf_ptr = argbuf; 
    array_ptr<int> buf_lo = argbuf;
    array_ptr<int> buf_hi = end
    where bounds(buf_ptr) == (buf_lo, buf_hi);
    int sum = 0;
    while (buf_ptr < end) {
        array_ptr<int> tmp1 = buf_ptr
        assert(tmp1 >= buf_lo && tmp1 < buf_hi);
        sum += *tmp1;
        // array_view<int> tmp = buf + 1;
        array_ptr<int> tmp_ptr = buf_ptr + 1;
        array_ptr<int> tmp_lo = buf_lo;
        array_ptr<int> tmp_hi = buf_hi
        where bounds(tmp_ptr) == (tmp_lo, tmp_hi)
        buf_ptr = tmp_ptr;
        buf_lo = tmp_lo;
        buf_hi = tmp_hi
        where bounds(buf_ptr) == (buf_lo, buf_hi);
    }
    return sum;
}
\end{verbatim}

The compiler can then apply copy propagation and reverse copy
propagation:

\begin{verbatim}
int sum (array_ptr<int> argbuf where bounds(argbuf) == (argbuf, end), 
         array_ptr<int> end) {
    // buf = new array_view(argbuf, argbuf, end);
    array_ptr<int> buf_ptr = argbuf; 
    array_ptr<int> buf_lo = argbuf;
    array_ptr<int> buf_hi = end
    where bounds(buf_ptr) == (buf_lo, buf_hi);
    int sum = 0;
    while (buf_ptr < end) {
        assert(buf_ptr >= buf_lo && buf_ptr < buf_hi);
        sum += *buf_ptr;
        // array_view<int> tmp = buf + 1;
        buf_ptr = buf_ptr + 1;  // result of reverse copy prop
        buf_lo = buf_lo;
        buf_hi = buf_hi;
        where bound(buf_ptr) = (buf_lo, buf_hi);
    }
    return sum;

}
\end{verbatim}

It can then eliminate eliminating identity assignments:

\begin{verbatim}
int sum (array_ptr<int> argbuf where bounds(argbuf) == (argbuf, end), 
         array_ptr<int> end) {
    // buf = new array_view(argbuf, argbuf, end);
    array_ptr<int> buf_ptr = argbuf; 
    array_ptr<int> buf_lo = argbuf;
    array_ptr<int> buf_hi = end
    where bounds(buf_ptr) == (buf_lo, buf_hi);
    int sum = 0;
    while (buf_ptr < end) {
        assert(buf_ptr >= buf_lo && buf_ptr < buf_hi);
        sum += *buf_ptr;
        // array_view<int> tmp = buf + 1;
        buf_ptr = buf_ptr + 1
        where bound(buf_ptr) == (buf_lo, buf_hi)
    }
    return sum;
}
\end{verbatim}

It can do another round of copy propagation and dead-code elimination,
producing:

\begin{verbatim}
int sum(array_ptr<int> argbuf where bounds(argbuf) == (argbuf, end), 
        array_ptr<int> end) {
    // buf = new array_view(argbuf, argbuf, end);
    buf_ptr = argbuf;
    where bounds(buf_ptr) == (argbuf, end);
    int sum = 0;
    while (buf_ptr < end) {
        assert(buf_ptr >= argbuf && buf_ptr < end);
        sum += *buf_ptr;
        // array_view<int> tmp = buf + 1;
        buf_ptr = buf_ptr + 1
        where bound(buf_ptr) == (argbuf, end);
    }
    return sum;
}
\end{verbatim}

Finally, the compiler can analyze this code and realize that the assert
statement is redundant. The condition \texttt{(buf\_ptr \textless{}
end)} is redundant with the while condition. The variable
\texttt{buf\_ptr} always monotonically increases and is not subject to
overflow:
\begin{verbatim}
int sum(array_ptr<int> argbuf where bounds(argbuf) == (argbuf, end), 
        array_ptr<int> end) {
    // buf = new array_view(argbuf, argbuf, end);
    buf_ptr = argbuf;
    where bounds(buf_ptr) == (argbuf, end);
    int sum = 0;
    while (buf_ptr < end) {
        sum += *buf_ptr;
        // array_view<int> tmp = buf + 1;
        buf_ptr = buf_ptr + 1
        where bound(buf_ptr) == (argbuf, end);
    }
    return sum;
}
\end{verbatim}

The \keyword{where} clauses are static only and the compiler can erase
them from the final code:

\begin{verbatim}
int sum(array_ptr<int> argbuf where bounds(argbuf) == (argbuf, end), 
        array_ptr<int> end) {
    // buf = new array_view(argbuf, argbuf, end);
    buf_ptr = argbuf;
    int sum = 0;
    while (buf_ptr < end) {
        sum += *buf_ptr;
        // array_view<int> tmp = buf + 1;

        buf_ptr = buf_ptr + 1
    }
    return sum;
}
\end{verbatim}