% !Tex root = checkedc.tex

\chapter{Bounds declarations for structure types}
\label{chapter:structure-bounds}

This chapter extends reasoning about bounds to objects with structure
types by introducing bounds declarations for structure members.
Structure types may have members with \arrayptr\ types. Those
members must have \emph{member bounds declarations} associated with them in order for
the values stored there to be used to access memory.

The declarations of
structure members may include \keyword{where} clauses that declare member
bounds. The member bounds declarations may also be placed inline.
A structure declaration may also include \keyword{where} clauses at the same
level as member declarations.  Here are examples:

\begin{lstlisting}
struct S {
    array_ptr<int> data where data : count(num);
    int num;
};
\end{lstlisting}
or
\begin{lstlisting}
struct S {
    array_ptr<int> data : count(num);
    int num;
};
\end{lstlisting}
or
\begin{lstlisting}
struct S {
    array_ptr<int> data;
    int num;
    where data : count(num);
};
\end{lstlisting}

Member bounds declarations are program invariants that are assumed to be true
by default for objects of that type. A member bounds declaraton may be suspended for a specific
object to allow for initialization of the object or modification of the
members involved in the member bounds declarations. The member bounds
declaration is declared to hold again after the initialization or modification is done.
Here is an example of variable of type \code{S} being initialized:

\begin{lstlisting}
void f(int len) {
    S y 
    where suspends(y.data);
    array_ptr<int> newarr : count(len) = malloc(sizeof(int) * len);
    y.data = newarr;
    y.num = len
    where holds(y.data);    // the member bounds for y.data now holds
    ...
}
\end{lstlisting}

\keyword{Suspends} and \keyword{holds} are dataflow-sensitive declarations of 
the states of the member bounds declarations for specific members of variables. They can be
applied to variable members that have member bounds.

They also can be applied to variables or variable members with structure
types as syntactic short-hand. In that case, they apply to the nested
members of the variable or variable member. The example could also be
written as:

\begin{lstlisting}
void f(int len) {
    S y 
    where suspends(y);
    array_ptr<int> newarr : count(len) = malloc(sizeof(int) * len))
    y.data = newarr;
    y.num = len
    where holds(y);    // the member bounds for y.data now holds
    ...
}
\end{lstlisting}

Making member bounds declarations be invariants provides a way to deal with issues
caused by aliasing. There can be pointers to data structures or members
of data structures. There may be multiple pointers to a single
structure object in memory. When there are multiple pointers,
the pointers are called aliases because they all name the same
memory location.  Aliasing makes it hard to reason about programs.

Consider the example:
\begin{lstlisting}
f(S *q, S *r, bool b)
{
   if (b) {
      q->arr = malloc(sizeof(int)*5);
      q->len = 5;
   }
   else {
      r->arr = malloc(sizeof(int)*5);
      r->len = 5;
   }
}
\end{lstlisting}

Even when b is true, the value \code{r->arr} may still be
changed by a call to f. This can happen when \code{q} and \code{r}
are the same value and are aliases for the same memory location.
Changing one named value (\code{q->arr}) can have the
effect of changing some other value with a distinct name
(\code{r->arr}). In general, it is difficult to know
whether an assignment through one pointer variable is affecting the
members of other pointer variables.

Member bounds declarations being program invariants for structure members allows
localized reasoning about the members. A programmer can assume that the
bounds declarations are true for members of objects of that type. The member bounds
declaration may be suspended temporarily for specific objects while they are being
initialized or modified.

The bounds declarations for variables with structure types and the
\keyword{suspends} and \keyword{holds} declarations will be checked statically
for validity.  Section~\ref{section:checking-bounds-with-structures} extends
the rules from Chapter~\ref{chapter:checking-bounds} to variables with structure
types and to \keyword{suspends} and \keyword{holds} declarations.

In the rest of this chapter, to simplify the description, assumptions
about address-taken variables similar to those in
Section~\ref{section:bounds-declarations} are made.
It is assumed that none of the variables or members of variables on the
left-hand side of bounds-declarations have their addresses taken. It is
assumed also that the values of variables or members of variables whose
addresses are taken are not used in bounds expressions.

\section{Declaring bounds for structure members}

Member bounds declarations have the form:
\begin{tabbing}
\var{member}\=\var{-bounds-decl:}\\
\> \var{member-path} \code{:} \var{member-bounds-exp} \\
\\
\var{member-path:}\\
\> \var{identifier} \\
\> \var{member-path} \code{.} \var{identifier}
\end{tabbing}

A member path is a sequence of one or more member names, separated by
the `\code{.}' operator. The sequence of members must be a valid sequence of
member accesses for the structure type. The common case of using a
member of the structure type is simply a member path of length 1.

Member bounds expressions are similar to the bounds expressions
described in Section~\ref{section:bounds-declarations}, 
except that members of the structure type are
used in place of variables in the non-modifying expressions. In
addition, pointer indirection and indirect member references are
excluded.

A structure member whose type is \arrayptr\ or a
checked array type may have at most one bounds declared for it. The
typing rules for member bounds declarations are similar to those for
variable bounds declarations. For bounds declarations of the form
\boundsdecl{\var{member-path}}{\boundscount{\var{e1}}}, the
\var{member-path} cannot have the type \arrayptrvoid\ and
the expression \var{e1} must have an integral type. For bounds declarations
of the form \boundsdecl{\var{member-path}}{\bounds{\var{e1}}{\var{e2}}},
the types of \var{e1} and \var{e2} must be pointers to the same type.
Typically \var{member-path} is also  a pointer to that type or an
array of that type.  However, \var{member-path} can be a pointer to
or an array of a different type.

A structure consists of a list of member declarations, each of which
consists of a type specifier followed by one or more structure member
declarators. Structure member declarators are changed to allow 
optional in-line specification of member bounds and \keyword{where}
clauses.

\begin{tabbing}
\var{struct}\=\var{-member-declarator:}\\
\> \var{declarator where-clause\textsubscript{opt}} \\
\> \var{declarator\textsubscript{opt}} \code{:}
   \var{constant-expression} \\
\> \var{declarator} \code{:} \var{member-bounds-exp}
\var{where-clause\textsubscript{opt}} \\
\\
The list of member declarations is extended to include \keyword{where}
clauses:\\
\\
\var{struct-member-declaration:}\\
\> \ldots{} \\
\> \var{where-clause}\textsubscript{opt}  \\
\\
The remaining syntax for specifying a structure remains unchanged: \\
\\
\var{struct-or-union-specifier:}\\
\> \var{struct-or-union identifier\textsubscript{opt}} \lstinline|{|
\var{struct-member-declaration-list} \lstinline|}| \\
\\
\var{struct-member-declaration-list:} \\
\> \var{struct-member-declaration} \\
\> \var{struct-member-declaration-list struct-member-declaration} \\
\\
\var{struct-member-declaration:} \\
\> \var{specifier-qualifier-list struct-member-declarator-list} \code{;} \\
\\
\var{struct-member-declarator-list:} \\
\> \var{struct-member-declarator} \\
\> \var{struct-declarator-list} \code{,} \var{struct-member-declarator} 
\end{tabbing}

A member bounds expression can use members and child members of the
structure being declared. Any member paths occurring in the member
bounds expressions must start with members of the structure type being
declared. Here is an example of the use of child members:

\begin{lstlisting}
struct A {
   array_ptr<int> data;
};

struct N {
    int num;
};

struct S {
   A a
   where count(a.data) == n.num;
   N n;
};
\end{lstlisting}

Allowing member bounds to use nested members of members complicates
explaining concepts. Sometimes concepts will be explained using member
bounds that use only immediate members and then generalized to handle
nested members.

\section{Bounds declarations for variables with structure types}

To describe facts about members of specific variables, the left-hand
sides of bound declarations are generalized to allow members of
variables. The term \emph{variable member path} stands for variables or
variables with member accesses. Variable member paths are used where
variables were allowed:
\begin{tabbing}
\var{bounds}\=\var{-decl:}\\
\> \var{var-member-path} \code{:} \var{bounds-exp} \\
\\
\var{var-member-path:} \\
\> \var{identifier} \\
\> \var{var-member-path} \code{.} \var{identifier}
\end{tabbing}

The first identifier in a variable member path must be the name of a
variable. The rest of the path, if there is one, must
describe a member path for the structure type that has an
\arrayptr\ type. The member path can be used as a name for its
associated member bounds. Inline bounds declarations are still
restricted to variables.

\section{Declaring the state of member bounds declarations for variables}

Programmers may declare the state of a member bounds declaration
for a variable using two kinds of facts:
\begin{tabbing}
\var{fact:}\= \\
\> \var{\ldots{}} \\
\> \code{suspends(}\var{var-member-path}\code{)} \\
\> \code{holds(}\var{var-member-path}\code{)}
\end{tabbing}

If the \var{var-member-path} has an \arrayptr\ type, it must
have the form \var{x.path,} where \var{x} is a variable name. 
The fact \code{suspends}(\var{x\code{.}path}\code{)} means that the
member bounds declared by the type of \var{x} for \var{path} may not
hold for \var{x}. The fact \code{holds(}\var{x\code{.}path}\code{)}
means that the member bounds declared by the type of x for \var{path}
holds now for \var{x}.

As a convenient short-hand notation, the \var{var-member-path} can have
a structure type. In that case, the declaration applies to all member
bounds for the structure type and child members of the structure type.

\subsection{Parameters and return values}

The state of member bounds declarations can be declared for parameters and
return values using \keyword{suspends} and \keyword{holds} as well. By
default, the state is \code{holds}.

Consider the following structure type definitions:

\begin{lstlisting}
struct S {
    array_ptr<int> data : count(num);
    int num;
}

struct T {
    S arr1;
    S arr2;
    array_ptr<float> weights : count(len);
    int len;
}
\end{lstlisting}

Here are some function declarations involving the state of member
bounds:
\begin{lstlisting}
T f(T x where holds(x)) where holds(return_value)   // the default 

T f(T x where suspends(x.arr1))
where suspends(return_value.arr2)

T f(T x where suspends(x.arr1.data)
where suspends(return_value.arr1.data)
\end{lstlisting}

\subsection{Extent of declarations of member bounds state for variables}
\label{section:member-bounds-state-extent}

Declarations of the state of member bounds declarations are dataflow-sensitive
and follow rules similar to flow-sensitive bounds declarations.

We first define the set of state declarations that apply to a function
component, where a function component is a statement or
variable declaration.

The state declarations for variables and variable members with structure
types are expanded to state declarations of the individual members with
\arrayptr\ type. It is assumed that declarations of automatic
structure variables without initializers implicitly have \keyword{suspends}
declarations for the variables. All other declarations of structure
variables are assumed implicitly to have \keyword{holds} declarations for the
variables. The other declarations are either automatic variables with
initializers or variables with static storage, which are initialized to
0. Any \arrayptr\ members initialized to 0 have \boundsany, so
they satisfy their member bounds declarations.

For any declaration of member bounds state for \var{v}.\var{mp}, where
\var{v} is a variable and \var{mp} is a member path, if
\begin{enumerate}
\item
  There is some path from the declaration to the function component, and
\item
  \var{v} occurs in the function component, and
\item
  There is no other declaration of member bounds state for \var{v.mp}
  along the path
\end{enumerate}
then the declaration of member bounds state applies to the function
component.

Member bounds state declarations must be consistent and agree
along all paths to a function component.   If a variable occurring in a
function component has more than one state declaration that applies to
it at the component, then all the state declarations applying to it at
the component must be syntactically identical. This avoids ambiguity
about which state declaration applies to an occurrence of a variable.
It is an error for the member bounds state declarations to disagree.

The following example illustrates the declaration of member bounds
state. The structure \code{S} represents a variable length array,
where \code{data} holds a pointer to an array and \code{num} is the 
length of the array. The function \code{f} takes a length parameter  
\code{len} and creates an initialized instance of \code{S} in the 
variable \code{y}.  It then copies \code{y} to \code{z}.

\begin{lstlisting}
struct S {
   array_ptr<int> data
   where count(data)== num;
   int num;
};
\end{lstlisting}

Here is a version of \code{f} where all member bounds state declarations are
made explicit. For structure variable members whose member bounds are
suspended, the bounds declarations are made explicit as well.

\begin{lstlisting}
void f(int len) {
    S y where suspends(y.data);
    S z where suspends(z.data);
    int i, j;
    array_ptr<int> newarr : count(len) = malloc(sizeof(int) * len);
    y.data = newarr
    where y.data : count(len);
    y.num = len
    where holds(y.data);   // the member bounds for y.data now holds
    z = y
    where holds(z.data);
}
\end{lstlisting}

This can be written more succinctly as:

\begin{lstlisting}
void f(int len) {
    S y;
    S z;
    int i, j;
    array_ptr<int> newarr : count(len) = malloc(sizeof(int) * len);
    y.data = newarr
    where y.data : count(len);
    y.num = len
    where holds(y);   // y is initialized now
    z = y
    where holds(z);   // z is initialized now
}
\end{lstlisting}

\section{Integration of member bounds and bounds for variables}

A member of a structure variable may be covered by its member bounds
declaration and a bounds declaration at the same time. This coexistence
happens when a member is initialized to satisfy its member bounds
declaration. Here is a version of \code{f} that is annotated with both member
bounds and bounds declarations for variable members.
\begin{lstlisting}
void f(int len) {
    S y;
    array_ptr<int> newarr : count(len) = malloc(sizeof(int) * len);
    y.data = newarr
    where newarr : count(len) && y.data : count(len);
    y.num = len
    where holds(y.data) && y.data : count(len) && 
          y.data : count(y.num);
    ...
}
\end{lstlisting}

After the assignment \code{y.num = len}, the member bounds holds for
\code{y.data} and the bounds declaration \code{y.data : count(y.num)}
holds as well.

\subsection{Determining bounds for a use of an \arrayptr\ member of a variable}
\label{section:determining-variable-member-bounds}

When a member path \var{mp} of a variable \var{x} is used and
\var{x}\code{.}\var{mp} has type \arrayptr, the bounds for
\var{x}\code{.}\var{mp} are determined using these rules:

\begin{itemize}
\item
  If the use is within the extent of a bounds declaration
  \var{x}\code{.}\var{mp} \code{:} \var{bounds-exp} and
  \var{bounds-exp} is not \boundsunknown, \var{bounds-exp} is
  the bounds.
\item
  Otherwise, if the state of the member bounds for
  \var{x}\code{.}\var{mp} is \code{holds}, the member bounds for
  \var{x}\code{.}\var{mp} is used.
\item
  Otherwise, the bounds of \var{x}\code{.}\var{mp} is
  \boundsunknown.
\end{itemize}

\subsection{Suspends declarations and bounds for variables.}

When the member bounds declaration \var{mb} for a variable member is
suspended by a statement of the form:

\var{e2} \keyword{where} \code{suspends(}\var{x\code{.}mp}\code{);}

where \var{x} is a variable and \var{mp} is a member path, there is an
implicit bounds declaration for \var{x}\code{.}\var{mp} at the point
of suspension. This happens unless \var{e2} modifies a member \var{m}
of \var{x} that occurs in \var{mb}. In addition, the state of the
member bounds for \var{x}\code{.}\var{mp} must be \code{holds}
before the statement. The member bounds declaration \var{mb} is
converted to a bounds declaration by prefixing each occurrence of a
member path in \var{mb} with the expression ``\var{x}\code{.}''.

For example, given

\begin{lstlisting}
S copy_and_resize(S arg, int len) {
    array_ptr<int> newarr : count(len) = malloc(sizeof(int) * len)
       where suspends(arg);
    for (int i = 0; i < arg.num; i++) {
       newarr[i] = arg.data[i];
    }
    arg.data = newarr where arg.data : count(len);
    arg.num = len where arg.data : count(arg.num)
    where holds(arg);   // member bounds for arg holds now
    return arg;
}
\end{lstlisting}


There is implicitly a bounds declaration at the suspension of the member
bounds for arg:

\begin{lstlisting}
    array_ptr<int> newarr : count(len) = malloc(sizeof(int) * len)
       where suspends(arg) && arg.data : count(arg.num);
\end{lstlisting}

If the suspends were done after the assignment to \code{arg.data}:

\begin{lstlisting}
    arg.data = newarr
    suspends(arg);
\end{lstlisting}

there would not be an implicit bounds declaration because \code{arg.data} is
modified by the assignment.

At a declaration of a structure variable x, no implicit bounds
declarations are inserted if the declaration suspends the member bounds
for a member of x. There was no point at which the member bounds was
known to be true. For example,

\begin{lstlisting}
f(S arg where suspends(arg))
\end{lstlisting}

does not have an implicit bounds declaration of the form \lstinline|arg.data :count(arg.len)|.

\subsection{Holds declarations and bounds for variables}

At a \keyword{holds} declaration for a variable member path \var{x.mp},
the member bounds for \var{x.mp} must be implied by the bounds
declarations that are true about members of \var{x} and any facts that 
true at the point of the \keyword{holds} declaration.

\section{Bounds-safe interfaces}
\label{section:structure-bounds-safe-interfaces}

Just as existing functions can have bounds-safe interfaces declared for
them, existing structure types can have bounds-safe interfaces declared
for them. This allows checked code to use those data structures and for
the uses to be checked. Existing unchecked code is unchanged.
To create a bound-safe interface for a structure type, a programmer
declares member bounds or interface types for structure members with
unchecked pointer types.

Here is a member bounds declaration for a structure that is a counted buffer
of characters:

\begin{lstlisting}
struct CountedBuffer {
     char *arr : count(len);
     int len;
}
\end{lstlisting}

Here are bounds-safe interface types for members of a structure for binary
tree nodes. The structure contains pointers to two other nodes.  In
checked code, pointer arithmetic on those pointers is not allowed.

\begin{lstlisting}
struct BinaryNode {
    int data;
    BinaryNode *left : itype(ptr<BinaryNode>);
    BinaryNode *right : itype(ptr<BinaryNode>);
}
\end{lstlisting}

If bounds information is declared for one member of a structure with an
unchecked pointer type, it must be declared for all other members of the
structure with unchecked pointer types.

It is important to understand that the \emph{semantics of unchecked
pointers in unchecked contexts does not change even when bounds are declared 
for the pointers}. The declared bounds are used only by checked code that uses the
data structure, when storing checked pointers into the data structure, and when converting
unchecked pointers read from the data structure to checked pointers.  Code in unchecked
contexts that uses only unchecked pointer types is compiled as though the bounds-safe
interface has been stripped from the source code.

\section{Extending checking validity of bounds}
\label{section:checking-bounds-with-structures}

This section describes how to extend the checking in Chapter~\ref{chapter:tracking-bounds} 
to check variables with structure bounds and expressions with structure operations such 
as member assignment and member access.  This section is primarily of interest to compiler writers.

In Chapter~\ref{chapter:checking-bounds}, contexts map pointer variables to their bounds.
Contexts are extended to map structure variables to descriptions of bounds for their
members.  The bounds for members of a variable will be described using a set of pairs,
where the first element of each pair is a member path whose type is \arrayptr\
and the second element of each pair is a bounds expression.  The bounds expression may use 
variable member paths in addition to variables.

In Section~\ref{section:inferring-expression-bounds}, the inference of bounds for an
expression determines the bounds expression that applies to the value of the
expression.  For structure-typed expressions, inference is generalized to determine
the bounds expressions that apply to the \arrayptr\ members of the value 
of the expression.  This information is represented using the same
representation used for contexts: a set of pairs, where the first element 
is a member path whose type is \arrayptr\ and the second element is a 
bounds expression.  Given this representation, it is easy to define
the rules for inferring bounds for member accesses: a member access prunes the
set of pairs and shortens the member paths.

In Section~\ref{section:checking-assignment-expressions}, checking of assignment
expressions updates contexts.  The contexts are then used to check that expression
statements imply declared bounds.  They are also used to check expressions nested
within an expression that contain assignment expressions. 
Given the representation of contexts, the rules for updating contexts
are extended to update contexts pointwise for structure variable assignments and member 
assignments.  For a structure variable assignment, the entire set of pairs for a
variable is updated.  For an assignment to a member of a  structure variable,
only the set of entries  associated with that member are updated.  Both forms of assignment
invalidate bounds expressions that use variable member paths that are changed by the
assignment.

An expression statement is checked by determining the updated context for the
expression statement, determining the expected context for bounds after 
the statement, and then checking that the updated context implies the
validity of the bounds in the expected context (Section ~\ref{section:checking-expression-statements}).
For each variable in the expected context, it is checked that the bounds expression in
the updated context implies the expected bounds expression.  This is easily 
extended to a structure variable by checking for each member path for 
the variable that the bounds expression in the updated context implies
the expected bounds expression.

\subsection{Determining contexts}

The context for every statement can be determined by using dataflow analyses extended
pointwise to structure variables.  Each structure variable is expanded into the set
of variable member paths that represent all \arrayptr-typed variable member paths beginning
with the variable.  First, an iterative forward dataflow analysis is done to determine the
member bounds state at every point in a function, following the rules in
Section~\ref{section:member-bounds-state-extent}.  The analysis works on the set of variable
member paths.   It computes for each variable member path whether the member bounds are valid 
at each program point.  It is a compile-time error if the analysis determines that the state of
member bounds is inconsistent along different paths.

Second, a generalized version of the extent dataflow analysis of Section~\ref{section:computing-extent}
is done.  Each structure variable is expanded into the set of variable
member paths that represent all \arrayptr-typed variable member paths begining with 
the variable.  The analysis assigns one lattice value to each of the variable member path for
each program point in the function where the variable is in scope.

The context for a statement is computed using the results of the dataflow analyses.
For each structure variable in scope for the statement, the set of \arrayptr-typed variable member paths 
beginning at the variable is determined.  The set is mapped to a set of pairs as follows:
\begin{itemize}
\item For the first element, the structure variable is removed from the beginning of the
path to create a member path.
\item For the second element, the extent dataflow analysis is consulted for the \arrayptr\ member path.
   If the bounds expression is not \boundsunknown, it is used.   If it is \boundsunknown, the result
   of the dataflow analysis for the member bounds state is examined. If the member bounds
   state is valid, the member bounds is used.  It is transformed to use variable member
   bounds by prefixing the member paths that occur in it with the structure variable.
   Otherwise, \boundsunknown\ is used.
\end{itemize}

\subsection{Inferring bounds for expressions without assignments}

This section describes extensions to Section~\ref{section:inferring-expression-bounds} for
determining bounds.   The notation \boundsinfer{\var{e}}{\var{s}} is overloaded for expressions
with structure types to mean that \var{e} has a set of member paths with individual bounds expressions.
\begin{itemize}
\item Variables: If \var{x} has a structure type, the context is consulted for \var{x} for
the set \var{s} of pairs of member paths and bounds expressions.  \boundsinfer{\var{x}}{\var{s}}.
\item Member access: Given an expression \var{e}\code{.}\var{m}, where \var{m} is 
a member name and \boundsinfer{\var{e}}{\var{s}},
\begin{itemize}
\item If \var{e}\code{.}\var{m} has a structure type, then
\begin{itemize}
\item The set $\var{s}^{\prime}$ is defined as follows.  For each pair 
      \texttt{(\var{mp}, \var{b})} in \var{s}, if \var{mp} has the form 
      \var{m}\code{.}\var{rp}, then \texttt{(\var{rp}, \var{b})} is
      included in $\var{s}^\prime$.
\item  Given $\var{s}^\prime$, \boundsinfer{\var{e}\code{.}\var{m}}{$\var{s}^\prime$}.
\end{itemize}
\item Otherwise,
\begin{itemize}
\item If $\var{s} = \texttt{\{(\var{m}, \var{b})\}}$, then \boundsinfer{\var{e}\code{.}\var{m}}{b}
\item Otherwise, \boundsinfer{\var{e}\code{.}\var{m}}{\boundsunknown}
\end{itemize}
\end{itemize}
\item Function calls: {\em To be filled in}
\end{itemize}

\subsection{Assignment expressions}

{\em To be filled in.}

\section{Compatibility of structure types with bounds declarations}

The C Standard defines compatibility of two structure types declared in
separate translation units \cite[Section 6.2.7]{ISO2011}.  This definition
is extended to include member bounds declarations and member bounds-safe
interfaces.  If the structure types are completed in both translation
units, for each pair of corresponding members,
\begin{itemize}
\item If the members have unchecked pointer type,
\begin{itemize}
\item If the members both have bounds-safe interfaces, the bounds-safe
interfaces must either both be bounds expressions or both be interface
types. If both have bounds expressions, the bounds expressions must be
syntactically identical after being placed into a canonical form.
If both have interface types, the interface types must be compatible.
\item Otherwise,  one member must have a bounds-safe interface and the
other member must omit a bounds-safe interface, or both members must omit
bounds-safe interfaces.
\end{itemize}
\item Otherwise, both members must have member bounds declarations or both
members must not have member bounds declarations.  If both members have
member bounds declarations, the bounds expressions must be syntactically
identical after being placed into canonical form.
\end{itemize}
