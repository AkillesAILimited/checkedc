% !Tex root = checkedc.tex

\chapter{Bounds for structure types}
\label{chapter:structure-bounds}

This section extends reasoning about bounds to variables with structure
types by introducing bounds for members of structures.

Structure types may have members with \arrayptr\ types. Those
members must have \emph{member bounds} associated with them in order for
the values stored there to be used to access memory. A structure member
declaration may include a \keyword{where} clauses that declares member
bounds. The bounds for a member also may be placed inline at the
declaration of the member. The list of declarations of structure members
may also include where clauses at top-level. Here are examples:

\begin{verbatim}
struct S {
   array_ptr<int> data : count(num);
   int num;
};
\end{verbatim}

or

\begin{verbatim}
struct S {
    array_ptr<int> data where data : count(num);
    int num;
}
\end{verbatim}

or

\begin{verbatim}
struct S {
    array_ptr<int> data;
    int num;
    where data : count(num);
}
\end{verbatim}

Member bounds are invariants that are assumed to be true by default for
objects of that type. A member bound may be suspended for a specific
object to allow for initialization of the object or modification of the
members involved in the member bounds declarations. The member bound
must be declared to hold again before \arrayptr\ field covered
by the bounds can be used to access memory. Here is an example of
variable of type S being initialized:

\begin{verbatim}
void f(int len) {
    S y 
    where suspends(y.data);
    array_ptr<int> newarr : count(len) = (array_ptr<int>) malloc(sizeof(int) * len))
    y.data = newarr;
    y.num = len
    where holds(y.data);    // the member bounds for y.data now holds
    …
}
\end{verbatim}

Suspends and holds are dataflow-sensitive declarations of the state of
the member bounds for a specific member of variables. They can be
applied to variable members that have member bounds.

They also can be applied to variables or variable members with structure
type as a syntactic short-hand. In that case, they apply the nested
members of the variable or variable member. The example could also be
written as:

\begin{verbatim}
void f(int len) {
    S y 
    where suspends(y);
    array_ptr<int> newarr : count(len) = (array_ptr<int>) malloc(sizeof(int) * len))
    y.data = newarr;
    y.num = len
    where holds(y);    // the member bounds for y.data now holds
    ...
}
\end{verbatim}

Making member bounds be invariants provides a way to deal with issues
causing by aliasing. There can be pointers to data structures or members
of data structures. There may be multiple pointers to a single object or
member of an object stored in memory. When there are multiple pointers,
the pointers are said to be aliases because they all name the same
memory location. The object is aliased. Aliasing makes it hard to reason
about the behavior of the program.

Consider the example:
\begin{verbatim}
f(S *q, S *r, bool b)
{
   if (b) {
      q->arr = malloc(sizeof(int)*5);
      q->len = 5;
   }
   else {
      r->arr = malloc(sizeof(int)*5);
      r->len = 5;
   }
}
\end{verbatim}

Even when b is true, the value \texttt{r-\textgreater{}arr} may still be
changed by a call to f. This can happen when \texttt{q} and \texttt{r}
are the same value and are aliases for the same memory location.
Changing one named value (\texttt{q-\textgreater{}arr}) can have the
effect of changing some other value with a distinct name
(\texttt{r-\textgreater{}arr}). In general, it is difficult to know
whether an assignment through one pointer variable is affecting the
members of other pointer variables.

Member bounds being invariants for structure members allow localized
reasoning about bounds. A programmer can assume that the bounds are true
for objects of that type. The member bounds may be suspended temporarily
for specific objects while they are being initialized or modified.

The bounds declarations for variables with structure types and the
suspend and hold declarations will be checked statically for validity.
Chapter~\ref{chapter:checking-structure-bounds} extends the rules 
from Chapter~\ref{chapter:checking-bounds}
to variables with structure
types and to suspends and holds declarations.

In the rest of this section, to simplify the description, assumptions
about address-taken variables similar to those in
Section~\ref{section:bounds-declarations} are made.
It is assumed that none of the variables or members of variables on the
left-hand side of bounds-declarations have their addresses taken. It is
assumed also that the values of variables or members of variables whose
addresses are taken are not used in bounds expressions.

\section{Declaring bounds for structure members}

Member bounds declarations have the form:
\begin{tabbing}
\var{member}\={-bounds-decl:}\\
\> \var{member-path} : \var{member-bounds-exp} \\
\\
\var{member-path:}\\
\> \var{identifier} \\
\> \var{member-path} \texttt{.} \var{identifier}
\end{tabbing}

A member path is a sequence of one or more member names, separated by
the `.' operator. The sequence of members must be a valid sequence of
member accesses for the structure type. The common case of using a
member of the structure type is simply a member path of length 1.

Member bounds expressions are similar to the bounds expressions
described in Section~\ref{section:bounds-declarations}, 
except that members of the structure type are
used in place of variables in the non-modifying expressions. In
addition, pointer indirection and indirect member references are
excluded.

A structure member whose type is \arrayptr\ or an incomplete
checked array type may have at most one bounds declared for it. The
typing rules for member bounds declarations are the same as those for
bounds declarations. For bounds declarations of the form
\boundsdecl{\var{member-path}}{\boundscount{\var{e1}}}, the
member-path must have an \arrayptr\ type and cannot be the type
\arrayptrvoid. The expression
\var{e1} must have an integral type. For bounds declarations of the
form \boundsdecl{\var{member-path}}{\bounds{\var{e1}}{\var{e2}}}, 
the types of \var{x}, \var{e1}, and \var{e2}
must be \arrayptr\ types. The types must be the same type,
except that any of them can be
\arrayptrvoid.

A structure consists of a list of member declarations, each of which
consists of type specifier followed by one or more structure member
declarators. Structure member declarators are changed to allow an
optional where clause or in-line specification of member. The in-line
specification of member follows syntax that is already used for
declaring bitfields:

\begin{tabbing}
\var{struct}\=\var{-member-declarator:}\\
\> \var{declarator where-clause\textsubscript{opt}} \\
\> \var{declarator\textsubscript{opt}} \texttt{:}
   \var{constant-expression} \\
\> \var{declarator} \texttt{:} \var{member-bounds-exp}
\var{where-clause\textsubscript{opt}} \\
\\
The list of member declarations is extended to include \keyword{where}
clauses:\\
\\
\var{struct-member-declaration:}\\
\> \ldots{} \\
\> \var{where-clause}\textsubscript{opt}  \\
\\
The remaining syntax for specifying a structure remains unchanged: \\
\\
\var{struct-or-union-specifier:}\\
\> \var{struct-or-union identifier\textsubscript{opt}} \texttt{\{}
\var{struct-member-declaration-list} \texttt{\}} \\
\\
\var{struct-member-declaration-list:} \\
\> \var{struct-member-declaration} \\
\> \var{struct-member-declaration-list struct-member-declaration} \\
\\
\var{struct-member-declaration:} \\
\> \var{specifier-qualifier-list struct-declarator-list} \texttt{;} \\
\\
\var{struct-declarator-list:} \\
\> \var{struct-declarator} \\
\> \var{struct-declarator-list} \texttt{,} \var{struct-declarator} 
\end{tabbing}

A member bounds expression can use members and child members of the
structure being declared. Any member paths occurring in the member
bounds expressions must start with members of the structure type being
declared. Here is an example of the use of child members:

\begin{verbatim}
struct A {
   array_ptr<int> data;
}

struct N {
    int num;
}

struct S {
   A a
   where count(a.data) == n.num;
   N n;
};
\end{verbatim}

Allowing member bounds to use nested members of members complicates
explaining concepts. Sometimes concepts will be explained using member
bounds that use only immediate members and then generalized to handle
nested members.

\section{Bounds declarations for variables with structure types}

To describe facts about members of specific variables, the left-hand
sides of bound declarations are generalized to allow members of
variables. The term \emph{variable member path} stands for variables or
variables with member accesses. Variable member paths are used where
variables were allowed:
\begin{tabbing}
\var{bounds}\=\var{-decl:}\\
\> \var{var-member-path} \texttt{:} \var{bounds-exp} \\
\\
\var{var-member-path:} \\
\> \var{identifier} \\
\> \var{var-member-path} \texttt{.} \var{identifier}
\end{tabbing}

The first identifier in a variable member path must be the name of a
local variable or parameter. The rest of the path, if there is one, must
describe a member path for the structure type that has an
\arrayptr\ type. The member path can be used as a name for its
associated member bounds. Inline bounds declarations are still
restricted to variables.

\section{Declaring the state of member bounds for variables}

Programmers may declare the state of a member bounds for a variable
using two kinds of facts:
\begin{tabbing}
\var{fact:}\= \\
\> \var{\ldots{}} \\
\> \texttt{suspends(\var{var-member-path})} \\
\> \texttt{holds(\var{var-member-path})}
\end{tabbing}

If the \var{var-member-path} has an \arrayptr\ type, it must
have the form \var{x.path,} where x is local variable or parameter
name. The fact \texttt{suspends(}\var{x.path}\texttt{)} means that the
member bounds declared by the type of x for \var{path} does not
currently hold for x. The fact \texttt{holds(}\var{x.path}\texttt{)}
means that the member bounds declared by the type of x for \var{path}
holds now for x.

As a convenient short-hand notation, the \var{var-member-path} can have
a structure type. In that case, the declaration applies to all member
bounds for the structure type and child members of the structure type.

\subsection{Parameters and return values}

The state of members bounds can be declared for parameters and return
values using suspends and holds as well. By default, the state is
\texttt{holds}.

Consider the following structure type definitions:

\begin{verbatim}
struct S {
    array_ptr<int> data : count(num);
    int num;
}

struct T {
    S arr1;
    S arr2;
    array_ptr<float> weights : count(len);
    int len;
}
\end{verbatim}

Here are some function declarations involving the state of member
bounds:
\begin{verbatim}
T f(T x where holds(x)) where holds(return_value)   // the default 

T f(T x where suspends(x.arr1))
where suspends(return_value.arr2)

T f(T x where suspends(x.arr1.data)
where suspends(return_value.arr1.data)
\end{verbatim}

\begin{quote}
\emph{One thing to consider is whether to allow functions to be
parametric with respect to the state of member bounds. This might be
useful to describe a function that sets one of the members of a
structure type and does not affect the state of other members.}
\end{quote}

\subsection{Extent of declarations of member bounds state for variables}

Declarations of the state of member bounds are dataflow-sensitive and
follow rules similar to flow-sensitive bounds declarations.

We first define the set of state declarations that apply to a function
component, where a function component is an expression statement,
variable declaration, or part of a compound statement.

The state declarations for variables and variable members with structure
types are expanded to state declarations of the individual members with
\arrayptr\ type. It is assumed that declarations of automatic
structure variables without initializers implicitly have suspends
declarations for the variables. All other declarations of structure
variables are assumed implicit to have holds declarations for the
variables. The other declarations are either automatic variables with
initializers or variables with static storage, which are initialized to
0. Any array\_ptr members initialized to 0 have \boundsany, so
they satisfy their member bounds declarations.

For any declaration of member bounds state for \var{v}.\var{mp}, where
v is variable and \var{mp} is a member path, if

\begin{enumerate}
\item
  There is some path from the declaration to the function component, and
\item
  \var{v} occurs in the function component, and
\item
  There is no other declaration of member bounds state for \var{v.mp}
  along the path
\end{enumerate}

then

\begin{enumerate}
\item
  The declaration of member bounds state applies
\end{enumerate}

This condition ensures the consistency of member bounds state
declarations for variables:

\begin{enumerate}
\item
  \emph{Agreement of bounds declarations}: If a variable occurring in a
  function component has more than one state declaration that applies to
  it at the component, then all the state declarations applying to it at
  the component must be syntactically identical. This avoids ambiguity
  about which state declaration applies to an occurrence of a variable.
  It an error for the member bounds state declarations to disagree.
\end{enumerate}

The following example illustrates the declaration of member bounds
state. The structure S represents variable length array, where data
holds a pointer to the data for the array and num is the length of the
array. The function f takes a length parameter len and creates an
initialized instance of S in the variable y. It then copies y to z.

\begin{verbatim}
struct S {
   array_ptr<int> data
   where count(data)== num;
   int num;
};
\end{verbatim}

Here is a version of f where all member bounds state declarations are
explicit. For structure variable members whose member bounds is
suspends, the bounds declarations are explicit as well.

\begin{verbatim}
void f(int len) {
    S y where suspends(y.data);
    S z where suspends(z.data);
    int i, j;
    array_ptr<int> newarr : count(len) = (array_ptr<int>) malloc(sizeof(int) * len))
    y.data = newarr
    where y.data : count(len);
    y.num = len
    where holds(y.data);   // the member bounds for y.data now holds
    z = y
    where holds(z.data);
}
\end{verbatim}

This can be written more succinctly as:

\begin{verbatim}
void f(int len) {
    S y;
    S z;
    int i, j;
    array_ptr<int> newarr : count(len) = (array_ptr<int>) malloc(sizeof(int) * len))
    y.data = newarr
    where y.data : count(len);
    y.num = len
    where holds(y);   // y is initialized now
    z = y
    where holds(z);   // z is initialized now
}
\end{verbatim}

\section{Integration of member bounds and bounds for variables}

A member of a structure variable may be covered by its member bounds
declaration and a bounds declaration at the same time. This coexistence
happens when a member is initialized to satisfy its member bounds
declaration. Here is a version of f that is annotated with both member
bounds and bounds declarations for members of a variable.
\begin{verbatim}
void f(int len) {
    S y;
    array_ptr<int> newarr : count(len) = (array_ptr<int>) malloc(sizeof(int) * len))
    y.data = newarr
    where newarr : count(len) && y.data : count(len)
    y.num = len
    where holds(y.data) && y.data : count(len) && 
          y.data : count(y.num)
    ...
}
\end{verbatim}

After the assignment \texttt{y.num = len}, the member bounds holds for
y.data and so does the bounds \texttt{y.data : count(y.num)}.

\subsection{Determining bounds for a use of an \arrayptr\ member of a variable}
\label{section:determining-variable-member-bounds}

When a member path \var{mp} of a variable \var{x} is used and
\var{x}\texttt{.}\var{mp} has type \arrayptr, the bounds for
\var{x}\texttt{.}\var{mp} are determined using these rules:

\begin{itemize}
\item
  If the use is within the extent of a bounds declaration
  \var{x}\texttt{.}\var{mp} \texttt{:} \var{bounds-exp} and
  \var{bounds-exp} is not \boundsnone, \var{bounds-exp} is
  the bounds.
\item
  Otherwise, if the state of the member bounds for
  \var{x}\texttt{.}\var{mp} is \texttt{holds}, the member bounds for
  \var{x}\texttt{.}\var{mp} is used.
\item
  Otherwise, the bounds of \var{x}\texttt{.}\var{mp} is
  \boundsnone.
\end{itemize}

\subsection{Suspends declarations and bounds for variables.}

When the member bounds \var{m}b for a variable member is suspended by a
statement of the form:

\var{e2} \keyword{where} \texttt{suspends(}\var{x.mp}\texttt{);}

where \var{x} is a variable and \var{mp} is a member path, there is an
implicit bounds declaration for \var{x}\texttt{.}\var{mp} at the point
of suspension. This happens unless \var{e2} modifies a member \var{m}
of \var{x} that occurs in \var{mb}. In addition, the state of the
member bounds for \var{x}\texttt{.}\var{mp} must be \texttt{holds}
before the statement. The member bounds declaration \var{mb} is
converted to a bounds declaration by prefixing each occurrence of a
member path in \var{mb} with the expression ``\var{x}\texttt{.}''.

For example, given

\begin{verbatim}
 S  copy_and_resize(S arg, int len) {
    array_ptr<int> newarr : count(len) = (array_ptr<int>) malloc(sizeof(int) * len)
       where suspends(arg);
    for (int i = 0; i < arg.num; i++) {
       newarr[i] = arg.data[i];
    }
    arg.data = newarr where arg.data : count(len);
    arg.num = len where arg.data : count(arg.num)
    where holds(arg);   // member bounds for arg holds now
    return arg;
}
\end{verbatim}


There is implicitly a bounds declaration at the suspension of the member
bounds for arg:

\begin{verbatim}
    array_ptr<int> newarr : count(len) = (array_ptr<int>) malloc(sizeof(int) * len)
       where suspends(arg) && arg.data : count(arg.num);
\end{verbatim}

If the suspends were done after the assignment to \texttt{arg.data}:

\begin{verbatim}
    arg.data = newarr
    suspends(arg);
\end{verbatim}

there would not be an implicit bounds declaration because arg.data is
modified by the assignment.

At a declaration of a structure variable x, no implicit bounds
declarations are inserted if the declaration suspends the member bounds
for a member of x. There was no point at which the member bounds was
known to be true. For example,

\begin{verbatim}
f(S arg where suspends(arg))
\end{verbatim}

does not have an implicit bounds declaration of the form \verb|arg.data :count(arg.len)|.

\subsection{Holds declarations and bounds for variables}\

At a holds declaration for a member path mp of a structure variable x,
the bounds declarations that are true about members of x, as well as any
facts that are true at the point of the holds declaration, must imply
that the member bounds for mp holds.

\begin{quote}
\emph{To do: this is just a special case of the fact that a member
bounds for a structure variable must be true after all statements in the
scope of the variable, unless the member bounds is suspended after the
statement. }

\emph{To do: it seems that requiring programmers to explicitly call out
interesting facts involving equality of variables or variable members
leads to a lot of verbosity in programs. We should reconsider whether
these must be explicitly declared by the programmer.}
\end{quote}

\section{Bounds-safe interfaces}
\label{section:structure-bounds-safe-interfaces}

Just as existing functions can have bounds-safe interfaces declared for
them, existing structure types can have bounds-safe interfaces declared
for them. This allows safe code to use those data structures and for
those uses to be checked. Existing unsafe code is unchanged.

To create a bound-safe interface for a structure type, a programmer
declares member bounds for structure members with unsafe pointer types.
A programmer may also declare whether an unsafe pointer should be
treated as \ptr\ type by safe code.

Here is a bound-safe interface for a structure that is a counted buffer
of characters:

\begin{verbatim}
struct CountedBuffer {
     char *arr : count(len);
     int len;
}
\end{verbatim}

Here is a bounds-safe for interface for a structure for a node in a
binary tree. The node contains two pointers to two other nodes. Pointer
arithmetic on those two other nodes is not allowed.

\begin{verbatim}
struct BinaryNode {
    int data;
    BinaryNode *left : ptr;
    BinaryNode *right : ptr;
}
\end{verbatim}

If bounds information is declared for one member of a structure with an
unsafe pointer type, it must be declared for all other members of the
structure with unsafe pointer types.

It is important to understand that the \emph{semantics of unsafe
pointers does not change even when bounds are declared for the
pointers}. The declared bounds are used only by safe code that uses the
data structure, when storing safe pointers into the data structure, and
when converting unsafe pointers data structures, and when reading safe
pointers from the data structure unsafe functions or uses unsafe data
structures.

\section{Bounds for pointers with information stored in unused bits}

C programs sometimes store information in unused bits of pointers. This
is typically done to save space in data structure. For example, a
program might store a tag in the least significant bits or most
significant bits of some pointers, knowing that those bits are always
zero otherwise. This section covers how to represent the bounds of these
pointers and extends the rules for checking bounds declarations to
expressions that modify bits of pointers.

The bounds of such a pointer are easy to described: they are the bounds
of the original pointer. If a pointer variable or member is used in its
own bounds, the additional information must be removed first. This is
done by masking out the bits that always 0.

For example, suppose there is the following structure:

\begin{verbatim}
struct S {
   array_ptr<int> p : bounds(p, p + 4);
   int tag : 2;
}
\end{verbatim}

A programmer might know that on a machine with 32 bit pointers, the
least 2 significant bits of p are always 0, so tag could be stored
within p. The bounds would be changed to:

\begin{verbatim}
struct S {
   array_ptr<int> p : bounds((array_ptr<int>) ((size_t) p & ~3), 
                             (array_ptr<int>) ((size_t) p & ~3) + 4) rel_align(char);
}
\end{verbatim}

There is one important caveat: tags cannot be stored in null pointers. A
null pointer may have bounds that are not valid. This is fine because
the null pointer may not be used to access memory. However, tagging a
null pointer could result in a non-null pointer with bounds that are not
valid. This non-null pointer could be used to incorrectly access memory.

\subsection{Extending bounds checking rules to integral
expressions}

C programs modify bits in pointers by converting pointers to integers,
operating on integers, and converting the integers back to pointers. The
rules for inferring bounds for expressions are extended to track bounds
through integral expressions:

\begin{itemize}
\item
  Given e1 op e2, where op is an additive, multiplicative, or bitwise
  binary operator on integral types, the bounds of e1 and e2 can be
  computed:

  \begin{itemize}
  \item
    If e1 has bounds b, where b is not bounds(none), and e2 has
    bounds(none), and it can be proved that e1 != 0, then e1 op e2
    \texttt{:-} b.
  \item
    If the special variable \texttt{expr\_current\_value} occurs in b

    \begin{itemize}
    \item
      If e1 op e2 is invertible, then inverse of e1 op e2 is substituted
      for \texttt{expr\_current\_value}.
    \item
      Otherwise the bounds of the expression are altered to be
      bounds(none).
    \end{itemize}
  \item
    Similar rules apply for the reverse situation where e1 has
    bounds(none) and e2 has bounds b, where b is not bounds(none).
  \end{itemize}
\item
  Given e1 op e2 where op is a shift operator, e1 is the integral value
  being shifted, and e2 is the shift amount,

  \begin{itemize}
  \item
    If e1 has bounds b, where b is not bounds(none), and e2 has
    bounds(none), and it can be proved that e1 != 0, then e1 op e2
    \texttt{:-} b.
  \item
    If the special variable \texttt{expr\_current\_value} occurs in b,
    bounds of the expression are altered to be bounds(none). Shift
    expressions are not invertible because they lose information.
  \end{itemize}
\end{itemize}

\subsection{An example}

The following code shows a tag of 1 can be set in a non-null pointer:

\begin{verbatim}
#define untagged_bounds(x) \
   bounds((array_ptr<int>) ((size_t) x & ~3), \
          (array_ptr<int>) ((size_t) x & ~3) + 4) rel_align(char)

S set(S s) 
{
  array_ptr<int> a : untagged_bounds(a) = s.p;
  if (a != NULL) {
    size_t y : untagged_bounds(a) = ((size_t) a) | 1;
    dynamic_assert(y & ~0x3 == (size_t) a & ~0x3); // always true
    // legal by substitution of y & 0x3 for a & ~0x3
    where y : object_bounds(y & ~0x3, y & ~0x3) relative_align(none);
    // exp : object_bounds(a,b) implies exp : bounds(a,b), no matter what
    // the value of exp
    where y : bounds(y ~0x3, y & ~0x3);
    // satifies member bounds requirements
    s.p = (array_ptr<int>) y;
   }
}
\end{verbatim}

This could be condensed to:
\begin{verbatim}
S set(S s) 
{
   array_ptr<int> a : bounds(a & ~0x3, a & ~0x3) = s.p;
   if (a != NULL) {
       size_t y : bounds(a & ~0x3, a & ~0x3) rel_align(char) = ((size_t) a) | 1;
       dynamic_assert(y & ~0x3 == a & ~0x3);
       where y : object_bounds(y & ~0x3, y & ~0x3) rel_align(char);
       s.p = (array_ptr<int>) y;
   }
} 
\end{verbatim}