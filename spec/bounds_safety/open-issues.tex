% !Tex root = checkedc.tex

\chapter{Pending work and open issues}
\label{chapter:open-issues}

This chapter discuss work that needs to be done and open issues.

\section{Pending work}

\begin{itemize}
\item The definition of \arrayviewT\ need to be extended to include
      relative alignment
\item Add a way to lift bounds to pointers to \arrayptr\ pointers
      and other lvalues used in their bounds.  This will be 
      necessary for handling pass-by-reference parameters.
      In C, pass-by-reference parameters are represented pointers
      to storage that holds the resulting values.  They are
      used to return multiple values from a function.
      For example, one parameter may be a pointer to an 
      \arrayptr\ and another parameter may be a pointer to an 
      integer that is the length of the \arrayptr:
\begin{verbatim}
f(ptr<array_ptr<int>> ptrbuf, ptr<int> ptrlen) {
   ...
}
\end{verbatim}
       There needs to be a way to describe the bounds for
       \texttt{*ptrbuf} using \texttt{*ptrlen}.
       
      This will also be necessary for handling address-of operations
      involving structure members with member bounds.
      Suppose \texttt{s} is a structure variable where the structure 
      has a member bounds. If the address of a member of \texttt{s} is
      taken and the member appears in the member bounds, the member
      bounds becomes a constraint on that pointer.  In particular, 
      there are constraints on values that may be written to memory using 
      that pointer.
\item Add a way to describe that safe pointers are not aliases
      of each other. It will be important to know that a pointer store 
      does not affect a pointer involved in a constraint.  For example,
      in \texttt{f}, \texttt{*ptrbuf} and \texttt{ptrlen} cannot alias 
      the same memory. Otherwise a store into an element of \texttt{*ptrbuf}
      might accidentally  modify the length of the buffer.
\item Allow conditional bounds expressions.   While conditional
expressions are allowed in the non-modifying expressions in a bounds expression, 
there is not a conditional form of bounds expressions.  This is useful for
specifying that an expression only has bounds if some condition is true.
This could be provided by adding a clause to bounds-exp that uses the
same syntax as C conditional expressions:
\begin{tabbing}
\var{bounds}\=\var{-exp:} \\
\>\texttt{\var{non-modifying-exp} ? \var{bounds-exp} : \var{bounds-exp}}
\end{tabbing}

We would need to add descriptions of introduction and elimination
forms, as well as enhance the rules for checking the validity of bounds.
\item Allow bounds declarations to declare bounds for expressions.
This would allow us to easily describe the bounds when pointers are tagged.
The bounds for the untagged pointer would be described. 
A tagged pointer could not be used to access memory.  It would have to be
untagged first  to have valid bounds.
\end{itemize}

\section{Language and library features to be addressed}

\begin{itemize}
\item
  Decide what to about null terminated arrays. Do we have special rules
  for them?
\item
  Old-style function declarations where argument list length or
  parameter/argument types could be mismatched at compile time, leading
  to undefined behavior.
\item
  Function pointers: the where clauses must become part of the signature
  of the pointer type.  In addition, we need to consider how implicit
  conversions of bounds-safe interfaces work with functino pointers.
\item
  Variable arguments
\item
  Pointer casts that produce incorrectly aligned pointers have undefined
  behavior, according to the C11 standard. This hole should be filled in
  for safe pointer types. For safe pointer types, we should specify
  either (1) dereferencing an incorrectly aligned pointer shall cause a
  runtime error or (2) the cast itself shall check any alignment
  requirements. For case 1, note that safe pointer arithmetic is already
  defined to preserve misalignment.
\end{itemize}

\section{Concrete syntax}
 
The current syntax for describing post-conditions places a where clause
after the function parameter list declaration:

\begin{verbatim}
f( ...)
where cond1 ...
\end{verbatim}

This syntax might lead to confusion. We might want to adopt an alternate
syntax that makes this clearer. Some suggestions are the keywords
\texttt{on\_return} or \texttt{after}:

\begin{verbatim}
f(...)
on_return cond1 ...

f(...)
after cond1 ...
\end{verbatim}

Section~\ref{section:alternate-pointer-type-syntax} describes alternate
syntax proposals for pointer types.
