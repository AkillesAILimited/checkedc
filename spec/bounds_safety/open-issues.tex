% !Tex root = checkedc.tex

\chapter{Pending work and open issues}
\label{chapter:open-issues}

This chapter discuss work that needs to be done and open issues.

\section{Pending work}

\begin{itemize}
\item Allow conditional bounds expressions.   While we allow conditional
expressions in the non-modifying expressions in a bounds expression, we
do not have a conditional form of bounds expressions.  This is useful for
specifying that an expression only has bounds if some condition is true.
This could be provided by adding a clause to bounds-exp that uses the
same syntax as C conditional expressions:
\begin{tabbing}
\var{bounds}\=\var{-exp:} \\
\>\texttt{\var{non-modifying-exp} ? \var{bounds-exp} : \var{bounds-exp}}
\end{tabbing}

We have to add descriptions of introduction and elimination
forms, as well as enhance the rules for checking the validity of bounds.
\item Allow bounds declarations to declare bounds for expressions.
This would allow us to easily describe the bounds when pointers are tagged.
We would describe the bounds for the untagged pointer.  A tagged pointer
could not be used to access memory.  It would have to be untagged first 
to have valid bounds.
\end{itemize}

\section{Language and library features to be addressed}

\begin{itemize}
\item
  Decide what to about null terminated arrays. Do we have special rules
  for them?
\item
  Old-style function declarations where argument list length or
  parameter/argument types could be mismatched at compile time, leading
  to undefined behavior.
\item
  Function pointers: the where clauses must become part of the signature
  of the pointer type.
\item
  Variable arguments
\item
  Pointer casts that produce incorrectly aligned pointers have undefined
  behavior, according to the C11 standard. This hole should be filled in
  for safe pointer types. For safe pointer types, we should specify
  either (1) dereferencing an incorrectly aligned pointer shall cause a
  runtime error or (2) the cast itself shall check any alignment
  requirements. For case 1, note that safe pointer arithmetic is already
  defined to preserve misalignment.
\end{itemize}

\section{Bounds and null values}

It is troublesome that the current meaning of bounds expressions is
conditional on values not being null. 
From Section~\ref{section:bounds-declarations}:

\begin{quote}
The meaning of a bounds expression can be defined more precisely. At
runtime, given an expression \var{e} with a bounds expression
\texttt{bounds(}\var{lb}\texttt{,} \var{ub}\texttt{)}, let the runtime
values of \var{e}, \var{lb}, and \var{ub} be \var{ev}, \var{lbv},
and \var{ubv}, respectively. The value ev will be \texttt{NULL} or have
been derived via a sequence of operations from a pointer to some object
\var{obj} with \texttt{bounds(}\var{low}\texttt{,}
\var{high}\texttt{)}. The following statement will be true at runtime:
\var{ev} \texttt{== 0 \textbar{}\textbar{} (}\var{low}
\texttt{<=} \var{lbv} \texttt{\&\&} ubv \texttt{<=}
\var{high}\texttt{)}. In other words, if \var{ev} is null, the bounds
may or may not be valid. If \var{ev} is non-null, the bounds must be
valid. This implies that any access to memory where \var{ev} \texttt{!=
0 \&\&} lbv \texttt{<=} \var{ev} \texttt{\&\&} \var{ev}
\texttt{<} \var{ubv} will be within the bounds of \var{obj}.
\end{quote}

The problem with this is that it means that e: bounds(lb, ub) does not
imply e + 4 : bounds(lb, ub). The counter-example is the runtime value
of e being 0. Let ev be the runtime value of e and ev == 0. For the
first expression, ev == 0 \textbar{}\textbar{} (low <= lbv
\&\& ubv <= high) must be true. For the second expression ev +
4 == 0 \textbar{}\textbar{} (low <= lbv \&\& ubv <=
high) must be true. This means the first expression must imply that (low
<= lbv \&\& ubv <= high) is true. It does not: (A
\textbar{}\textbar{} B) being true does not imply B.

We handle this by forbidding pointer arithmetic involving null pointers,
which requires runtime checking. However, this breaks down in the
presence of casts between integers and pointer and bitwise manipulation
of pointers. Given a variable \arrayptrint\ x,
the following two expressions should be equivalent:

\texttt{x + 1} and \texttt{(\arrayptrint) (((size\_t) x) + sizeof(int))}

When x is \texttt{0}, the first expression results in a runtime failure,
while the second does not.

This suggests that we have defined the meaning of bounds in a way such
that a ``core'' property is not captured. An alternate ``core''
definition of bounds would be:

\begin{quote}
The meaning of a bounds expression can be defined more precisely. At
runtime, given an expression \var{e} with a bounds expression
\texttt{object\_bounds(}\var{lb}\texttt{,} \var{ub}\texttt{)}, let the
runtime values of \var{e}, \var{lb}, and \var{ub} be \var{ev},
\var{lbv}, and \var{ubv}, respectively. The value ev have been derived
via a sequence of operations from a pointer to some object \var{obj}
with \texttt{object\_bounds(}\var{low}\texttt{,} \var{high}\texttt{)}.
The following statement will be true at runtime: \var{low}
\texttt{<=} \var{lbv} \texttt{\&\&} ubv \texttt{<=}
\var{high}. In other words, the bounds are always valid. This implies
that any access to memory where lbv \texttt{<=} \var{ev}
\texttt{\&\&} \var{ev} \texttt{<} \var{ubv} will be within
the bounds of \var{obj}.
\end{quote}

Then, given \texttt{object\_bounds(}\var{lb}\texttt{,}
\var{ub}\texttt{)}, we would define the meaning of \var{e} :
\texttt{bounds(}\var{lb},\var{ub}\texttt{)} as \var{e} \texttt{== 0
\textbar{}\textbar{}} \texttt{object\_bounds(}\var{lb},
\var{ub}\texttt{)}.

If we go down the path of using the introduction of
\texttt{object\_bounds} to eliminate the need to check for null pointer
arithmetic, it seems likely we will run into serious problems. It seems
like \texttt{bounds(}lb, ub\texttt{)} becomes hard (impossible?) to use
for functions that take in null pointer or a pointer to a valid object:

Consider:
\begin{verbatim}
f(array_ptr<int> x : bounds(x, x + 5) {
    array_ptr<int> y = x + 1;
    ...
}
\end{verbatim}

Let us suppose we want to declare bounds for \texttt{y}. The following
is now incorrect because \texttt{x + 1} does not fault when x is
\texttt{0} (null):

\begin{verbatim}
f(array_ptr<int> x : bounds(x, x + 5) {
    array_ptr<int> y : bounds(x, x + 5) = x + 1;
    ...
}
\end{verbatim}

To see why, suppose the runtime value of \texttt{x == 0}. The bounds
expression for x is valid at runtime: \texttt{x == 0
\textbar{}\textbar{} object\_bounds(0, 5)}. However, the bounds
expression for y is not valid. Its meaning is \texttt{1 == 0
\textbar{}\textbar{} object\_bounds(0, 5)} The first condition is false,
so the second condition \texttt{object\_bounds(0, 5)} must be true.
However, no object bounds(0, 5) has ever existed. y has not been derived
via a sequence of operations from a pointer to some object with
bounds(0, 5).

It is easier to see the problem by expanding the definition of bounds:

\begin{verbatim}
f(array_ptr<int> x : x == 0 || object_bounds(x, x + 5) {
    array_ptr<int> y : y == 0 || object_bounds(x, x + 5) = x + 1;
    ...
}
\end{verbatim}

There is another direction that seems promising, though:

\begin{itemize}
\item
  Still make safe pointer arithmetic that involving a runtime failure.
\item
  Introduce the notion of object\_bounds to handle the case where the
  bounds are known to represent bounds for a real object. This can be
  deduced from \var{e} : bounds(lb, ub) when \var{e} is non-null.
\item
  Once an expression has object\_bounds, any operation can be applied to
  the value of that expression and the resulting value still has the
  same object\_bounds annotation. We would need to have a version of
  bound expressions where relative alignment is not guaranteed.
\item
  Conversely, given \var{e} : object\_bounds(\var{lb}, \var{ub}),
  this can be used to deduce bounds(\var{lb}, \var{ub}) always.
\end{itemize}

This can be used in code that modifies bits of pointers, provided that
the pointers are non-null to start through. In general, it could be used
to track bounds through unsafe code. We could allow a pointer with
actual bounds to be converted to an integer, modified, and stored back
in a variable.

Suppose we add a modifier that allowed us to declare the
relative\_alignment that a pointer and its bounds satisfies. It could
either take a type or the keyword none. For example,
relative\_aligment(int) or relative\_alignment(none).

We could have code of the form:

\begin{verbatim}
array_ptr<int> x : bounds(x, x + 5) relative_alignment(none) = malloc(sizeof(int) * 5);
if (x ! = null) {
    array_ptr<int> y : object_bounds(x, x + 5) relative_alignment(none)  = x;
    size_t bit_pattern : object_bounds(y, y + 5) relative_alignment(none) = (size_t) x;
    bit_pattern |= 0xabcd;
    y = (array_ptr<int>) bit_pattern;
    ... *y ...  // checks that *y does not access outside of the bounds (x, x + 5);
}
\end{verbatim}

Suppose a clever programmer knows the lowest 2 bits of a pointer are
always 0 and wants to use them encode additional information about an
object:
\begin{verbatim}
struct S {
    array_ptr<int> m where m : count(10);
}
\end{verbatim}

Here is how the programmer can do this:
\begin{verbatim}
struct S  {
     array_ptr<int> m where m : bounds(m & ~0x3, m & ~0x3 + 10);
}
S set(S s) 
{
  array_ptr<int> a : bounds(a & ~0x3, a & ~0x3) relative_align(none) = s.m;
  if (a != NULL) {
    // legal because a != NULL
    where a : object_bounds(a & ~0x3, a & ~0x3) relative_align(none);
    size_t y : object_bounds(a & ~0x3, a & ~0x3) relative_align(none) = ((size_t) a) | 1;
    dynamic_assert(y & ~0x3 == a & ~0x3); // always true
    // legal by substitution of y & 0x3 for a & ~0x3
    where y : object_bounds(y & ~0x3, y & ~0x3) relative_align(none);
    // exp : object_bounds(a,b) implies exp : bounds(a,b), no matter what
    // the value of exp
    where y : bounds(y ~0x3, y & ~0x3);
    // satifies member bounds requirements
    s.m = (array_ptr<int>) y;
   }
}
\end{verbatim}

This could be condensed to:
\begin{verbatim}
S set(S s) 
{
   array_ptr<int> a : bounds(a & ~0x3, a & ~0x3) = s.m;
   if (a != NULL) {
       size_t y : object_bounds(a & ~0x3, a & ~0x3) relative_align(none) = ((size_t) a) | 1;
       dynamic_assert(y & ~0x3 == a & ~0x3);
       where y : object_bounds(y & ~0x3, y & ~0x3) relative_align(none);
       s.m = (array_ptr<int>) y;
   }
} 
\end{verbatim}

\section{Concrete syntax}

\subsection{Post-condition syntax}
The current syntax for describing post-conditions places a where clause
after the function parameter list declaration:

\begin{verbatim}
f( ...)
where cond1 ...
\end{verbatim}

This syntax might lead to confusion. We might want to adopt an alternate
syntax that makes this clearer. At the design discussion on 7/29/2015,
it was discovered that the group did not understand that the where
clause after the function parameter list applied to post-conditions.

Some suggestions are the keywords \texttt{on\_return} or \texttt{after}:

\begin{verbatim}
f(...)
where cond1 ...

f(...)
after cond1 ...
\end{verbatim}

\section{Discussion of optimizing bounds check and overflow}

We had the following discussion in Section~{section:bounds-checking-at-indirections}
that distracted from the
main point of the section. It's also confusing the overflow check for x
+ c is omitted. The point of the example below is that we can make code
efficient. That deserves a fuller discussion and should be buttressed by
empirical data.

Consider as an example, z = *(x+5); where x : bounds(x, x + c). The
compiler will produce code of the form

\begin{quote}
\begin{verbatim}
assert(x != null);
t1 = x + 5;
check x + 5 for overflow
assert(t1 != null && x <= t1 && t1 < x + c);
z = *t1;
\end{verbatim}
\end{quote}

This simplifies to:

\begin{quote}
\begin{verbatim}
assert(x != null);
t1 = x + 5;
check for overflow x + 5
assert(x <= t1 && t1 < x+c);
z = *t1;
\end{verbatim}
\end{quote}

The compiler can recognize that \texttt{x <= t1} is always true,
leading to:-

\begin{quote}
\begin{verbatim}
assert(x != null);
t1 = x + 5;
check for overflow of x + 5
assert(t1 < x + c);
z = *t1;
\end{verbatim}
\end{quote}