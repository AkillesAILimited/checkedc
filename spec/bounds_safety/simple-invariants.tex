% !Tex root = checkedc.tex

\chapter{Reasoning about bounds and simple program invariants}
\label{chapter:simple-invariants}

Programmers and compilers will need to reason about bounds as variables
are modified. In doing so, they will also need to reason about simple
program invariants. This section introduces extensions to C to support
reasoning about program invariants.

There are a variety of reasons why programmers and compilers may need to
reason about bounds:

\begin{itemize}
\item
  A programmer may wish to narrow the area of memory that can be
  accessed via an array pointer by modifying the bounds of the pointer.
\item
  A programmer may introduce a temporary variable and may wish to
  re-express existing bounds in terms of that new variable.
\item
  A programmer may write a loop that depends on a variable with bounds
  and also modifies one or more variables used in those bounds. Some
  simple reasoning may be needed to show that the loop body preserves
  the bounds.
\item
  A programmer may do a dynamic check to ensure that a bounds can be
  satisfied.
\item
  A compiler may need to check that the arguments to a function call
  satisfy the bounds requirements of the function parameters.
\item
  A programmer may want to write performance-critical code that is free
  of dynamic bounds checks. The programmer may introduce checks outside
  the performance-critical code that guarantee no dynamic checks are
  needed. The compiler may need to check that the bounds are statically
  satisfied in the performance-critical code.
\end{itemize}

We incorporate reasoning about bounds into the language because checking
the correctness of bounds statically is often the only practical choice.
If we were to ignore the correctness of the reasoning about bounds, we
would have to track the bounds for all pointer values dynamically to
ensure correct operation of programs. C programs often operate at such a
low level in the system that there may be no practical way to do this.

Because the reasoning about bounds is part of the language, compilers
will check the correctness of the reasoning, just as they check the
correctness of types. They must do so predictably and efficiently. Two
compilers for C with these extensions should always produce the same
answer. Compile times should be relatively unaffected and checking
should scale to code bases with millions of lines of code. This rules
out incorporating program verification techniques that are
heuristic-based or computationally expensive.

The computational limits for production compilers are severe. For
production compilers, compiler developers aim for algorithms that are
O(N) or (N lg N) time in terms of the size of a method or a program.
Compilers may encounter methods may have hundreds of thousands of lines
of code and whole programs that have millions of lines of code.
Production compiler developers use O(N\textsuperscript{2}) time
algorithms for individual functions only with great care, typically
including code to disable an algorithm, reduce precision, or switch to
an alternate algorithm that has better behavior at the expense of
producing worse code.

To handle the computational limits of compilers, we view the compiler's
job as one of proof checking. The compiler will check that individual
steps in a proof are correct. It will do limited inference in the
process of checking those steps. It will not try to infer which of the
many possible facts about a large method are relevant; the programmer
will need to state the relevant facts to use in inference.

\section{Facts about program points}

We start with facts about program points that a programmer may use to
reason about bound.  A  fact is a bounds declaration or a relational
statement about a variable and a non-modifying expression.
\begin{tabbing}
\var{fact:}\=\\
\>\var{bounds} \\
\>\var{variable relop non-modifying-exp} \\
\>\var{non-modifying-exp relop variable}
\end{tabbing}

Facts may be declared as part of \keyword{where} clauses.  
Chapter~\ref{chapter:tracking-bounds} included relational facts in
the grammar for \keyword{where} clauses but did not say
anything about them.  For expression statements, the compiler
will check that relational facts in the \keyword{where} clause are true 
using the facts true before the expression statement and the effect of
the expression. For a parameter, the \keyword{where} clause becomes a 
precondition that must be true at the call site. The compiler will check this precondition at the call sites. 

A \keyword{where} clause may be used as a stand-alone statement, 
in which case it stands on its own as a set of invariants and bounds
declaration:
\begin{verbatim}
where x>=0 && x<=10;
\end{verbatim}
This is just a special case of an expression statement
that omits the expression but still has \keyword{where} clause.

Facts are automatically inferred for the clauses of if-statements, 
the cases of switch statements, and from assignment statements. 
For if-statements, the test has to have
the form of a fact. For a switch statement, a fact is deduced from the
switch expression and the case statement.  For assignments, the
left-hand side must be a variable and the right-hand side must be
a non-modifying expression.

Here are some simple examples where the fact created by a control-flow
statement is explicitly declared using redundant \keyword{where} clauses:

\begin{verbatim}
if (x < 5) {
    where x < 5;
}
else {
    where x >= 5;
} 

switch (x) {
    case 0: {
        where x == 0;
        break;
    }
    case 1: {
        where x == 1;
        break;
     }   
    default: {
        where x != 0 && x != 1;
        break;
    }
}
\end{verbatim}

Here is a simple example that introduces the fact that \texttt{i} is always
\texttt{>=} 0:

\begin{verbatim}
int sum10(array_ptr<int> buf : count(10)) {
    int sum = 0;
    int i = 0
    where i >= 0;
    while (i<10) {
        sum += buf[i];
        i = i + 1
        where i >= 0;
    }
}
\end{verbatim}

From this fact, it can be deduced that the access to \texttt{buf} is always in
range. A \texttt{where} clause can be used double check this:

\begin{verbatim}
int sum10(array_ptr<int> buf : count(10)) {
    int sum = 0;
    int i = 0
    where i >= 0;
    while (i<10) {
        where (i>= 0 && i<10);
        sum += buf[i];
        i = i + 1
        where i >= 0;
    }
}
\end{verbatim}

\section{Checking facts}

The compiler checks facts in a \keyword{where} clause by gathering known
facts before an assignment statement and checking that the facts
declared in the \keyword{where} clause can be inferred (easily) from
known facts and the effect of the assignment statement. The compiler
finds known facts by examining each path from the beginning of the
function to the statement and identifying facts that are true along all
paths. Facts are introduced by \keyword{where} clauses, control-flow
statements, and simple assignments. Facts are removed by assignments to any
variable in a fact.

A programmer can find the facts that are available before an assignment
statement by looking at assignments that precede the statement. The
programmer usually only has to look at assignments to variables in the
\keyword{where} clause for the assignment statement. The programmer can then check
the facts along all paths from those assignments to the statement.

This analysis of facts available before a statement is similar to the
``available expression'' analysis done by optimizing compilers for
common-subexpression elimination. For common-subexpression elimination,
a compiler looks at all expressions. For analyzing available facts, a
compiler only looks at expressions in \keyword{where} clauses, expressions
inferred from control-flow statements, and simple assigments.

\subsection{Algorithm for checking correctness of facts}

To infer whether a fact is true in a \keyword{where} clause, the compiler first
computes the set of facts that are true after the expression statement.
We describe the common case where the expression is an assignment of the
form x:= e1.

\begin{enumerate}
\item
  If e1 is invertible and must have a well-defined value (cannot
  overflow or fails on overflow), the compiler takes the facts true
  before the statement and substitutes the inverse expression of e1 for
  any occurrences of x in the facts.
\item
  If e1 is not invertible or may not have a well-defined value, the
  compiler takes the facts before the statement and removes any facts
  where x occurs.
\item
  If e1 is a valid non-modifying expression, the compiler adds the fact
  x == e1 to the set of facts.
\end{enumerate}

For variables that are declared to be equal, the compiler chooses one of
them as the representative variable and substitutes it for all the other
variables. That is, given x == y and y == z, the compiler chooses one of
the variables (say, x) and substitutes it for y and z. The compiler
applies this substitution to both the set of facts and the fact that is
being checked.

The compiler then reduces the set of relational operators to the set
\texttt{<}, \texttt{<=}, \texttt{==},  and \texttt{!=}
by swapping operands and replacing \texttt{>} and \texttt{>=} 
with \texttt{<=} and \texttt{<} instead.

The compiler then puts each bounds declaration into a normal form so
that syntactic identity can be used to compare facts. For example,
operands for addition operations are commuted so that constant operands
appear first, followed by the remaining subexpressions in lexicographic
order. Integer subtraction a - b is turned into a + (-b). Additions
with only constant operands are simplified and algebraic identities for
addition and subtraction such as x + 0 are applied

The compiler next checks if the fact being checked is in the set of
facts. If it is not, it applies transitivity rules involving
\texttt{<}, \texttt{<=}, and \texttt{==}.
It also checks to see if \texttt{<}
implies \texttt{!=} is true, and if \texttt{<=} and
\texttt{>=} imply \texttt{==}.

\subsection{Bounds declarations and facts}

Bounds declarations are treated the same way as other facts. During the
checking process, the same simplifications that are applied to other
facts are applied to the right-hand side bounds expressions: the
compiler chooses representative variables and puts the non-modifying
expressions that make up a bounds expression into normal forms.

The rules for checking bounds declarations include rules for
transitivity. For \boundsdecl{\var{x}}{\bounds{\var{e1}}{\var{e2}}}, if
\texttt{\var{e1} <= \var{e3}}, then \var{e3} can
be substituted for \var{e1}. 
If \texttt{\var{e4} <= \var{e2}}, \var{e4} can be substituted for
\var{e2}.  For \boundsdecl{\var{x}}{\boundscount{\var{e1}}}, if \
\texttt{\var{e2} <= \var{e1} and \var{e2} >= 0}, 
\var{e2} can be substituted for \var{e1}.

\subsection{Pointer arithmetic and facts}

The rules used to check facts include the identities \texttt{\var{x} < \var{x} + \var{k}}
for positive \var{k} and \texttt{\var{x} + \var{k} < \var{k}} for
negative \var{k}, where \var{k} is a
constant and \var{x} is a variable that has a new pointer type (\ptr,
\arrayview, or \arrayptr). These identities are true
because pointer arithmetic overflow is defined as a runtime error for
new pointer types. This guarantees that adding \var{x} and \var{k} produces either
an in-range value or a runtime error (no value).

\subsection{Integer arithmetic and overflow}
\label{section:where-clauses-and-overflow}

Integral arithmetic may overflow or wraparound. A consequence of that is
that the rules used to check facts do not include the identities 
\texttt{\var{x} < \var{x} + \var{k}} 
for positive \var{k} and \texttt{\var{x} + \var{k} < \var{x}} for 
negative \var{k}, where \var{k} is a
constant and \var{x} has an integral type. These identities are not true in C.  For signed integer types, the computation \texttt{\var{x} + \var{k}}
may overflow.   According to C language
rules, program behavior is undefined in that case.  For unsigned
integer types, computations are performed modulo one plus
the maximum unsigned integer.

To use these identities for integers, the compiler needs to prove for
positive \var{k} that \texttt{\var{x} + \var{k} <= MAXINT} and for negative \var{k} that \texttt{MININT <= \var{x} + \var{k}}.  There are four rules that can be used to prove this:

\begin{itemize}
\item
  Given an integer \var{c} and a positive integer \var{k}, if 
  \texttt{\var{x} <= \var{c}} and \texttt{\var{c} + \var{k} <= MAXINT},
  then \texttt{\var{x} + \var{c} <= MAXINT}.
\item
  Conversely, given an integer \var{c} and a negative integer 
  \var{k}, if \texttt{\var{c} <= \var{x}} and \texttt{\var{c} +
   \var{k} >= MININT}, then \texttt{\var{x} + \var{k} >= MININT}
\item
  Given \texttt{\var{x} < \var{y}}, where \var{y} is any variable, 
  then \texttt{\var{x} + 1 <= MAXINT}.
\item
  Conversely, given \texttt{\var{x} > \var{y}}, where \var{y}
  is any variable, then \texttt{\var{x} - 1 >= MININT}.
\end{itemize}

\subsection{Checking function calls}

A function call is checked by substituting the actual parameters for the
formal parameters in the \keyword{where} clause for parameters.  The
resulting whre clauses must be checked to see if they are true, given
the facts true before the clause.  For the where clause for the returned
value, the actual parameters are substituted for the formal parameters.
The set of facts in the clause are then added to the set of facts true
after the call.

\subsection{Avoiding bounds checks at runtime}
\label{section:avoiding-bounds-checks}

It can be important to avoid dynamic bounds checks at runtime. The
built-in method \inbounds{\var{e}} can be used to do this. It can be applied
to expressions of type \arrayptr. At compile time, the compiler
checks that the bounds expressions for \var{e} are always true given the
facts true before the evaluation of \var{e}. At runtime, \inbounds{\var{e}}
simply returns the value of \var{e}. The compilers generates code for
\texttt{*\inbounds{\var{e}}} with no dynamic bounds checks. The prior example
can be written as:

\begin{verbatim}
int sum10(array_ptr<int> buf : count(10)) {
    int sum = 0;
    int i = 0
    where i >= 0;
    while (i<10) {
        sum += *in_bounds(buf + i);
        i = i + 1
        where i >= 0;
    }
}
\end{verbatim}

\section{Examples}

Here is a simple example of capturing the lower bounds of an
\arrayptr\ variable using another variable:

\begin{verbatim}
int sum(array_ptr<int> buf : bounds(buf, end), array_ptr<int> end) {
    array_ptr<int> tmp = buf;
    where buf : bounds(tmp, end); // substitute tmp for buf
    int sum = 0;
    while (buf < end) {
        sum += *buf;   
        buf = buf + 1; // buf bounds do not change, do not need to be redeclared
    }
    return sum;
}
\end{verbatim}

Here is a more complicated example where \texttt{buf} is incremented
\emph{and} \texttt{buf} is the lower bound:

\begin{verbatim}
int sum(array_ptr<int> buf : bounds(buf, end), array_ptr<int> end) {
    int sum = 0;
    while (buf < end) {
        sum += *buf;   
        buf = buf + 1
        where buf : bounds(buf, end);
    }
    return sum;
}
\end{verbatim}

Here are the steps that the compiler goes through, illustrated using 
\keyword{where} clauses. First, the compiler computes the facts true after
\texttt{buf = buf + 1}. The compiler computes the inverse expression for
\texttt{buf + 1}, which is \texttt{buf - 1}. It substitutes it into the
bounds expression that is true before the increment, producing the set
of facts after the increment of 
\texttt{\boundsdecl{buf}{\bounds{buf-1}{end}}}:

\begin{verbatim}
int sum(array_ptr<int> buf : bounds(buf, end), array_ptr<int> end) {
    int sum = 0;
    while (buf < end) {
        sum += *buf;   
        buf = buf + 1
        where buf : bounds(buf - 1, end); 
    }
    return sum;
}
\end{verbatim}

The next step is for the compiler to show that that \texttt{buf :
bounds(buf - 1, end)} implies \texttt{buf : bounds(buf, end)}. The
transitivity rule for bounds expressions implies that the compiler must
show \texttt{buf - 1 <= buf}. This follows from the identity
\texttt{\var{x} - \var{k} < \var{x}}, completing the validation of the bounds expression:

\begin{verbatim}
int sum(array_ptr<int> buf where bounds(buf) == (buf, end), 
        array_ptr<int> end) {
    int sum = 0;
    while (buf < end) {
        sum += *buf;   
        buf = buf + 1
        where bounds(buf) == (buf, end) 
    }
    return sum;
}
\end{verbatim}