% !Tex root = checkedc.tex

\chapter{Introduction}\label{introduction}

The C programming language \cite{Ritchie1988, ISO2011} allows programmers to use 
pointers directly. A pointer is an address of a location in memory. Programs may do
arithmetic on pointers, dereference them to read memory, or assign
through them to modify memory. The ability to use pointers directly
makes C well-suited for low-level systems programming that is ``close to
the hardware'' and allows programmers to write efficient programs. C
also unifies pointer types and array types. They can usually be used
interchangeably and array subscripting is a synonym for equivalent
pointer operations.

Pointers and the unification of arrays and pointers are one of the
strengths of the C programming language, allowing programmers to write
concise, efficient programs. At the same time, they are one of the main
sources of reliability and security problems in modern software. This is
because pointers and array indices are not bounds checked in C and
related languages such as C++. Bounds checking checks that a pointer or
array index is in bounds before it is used to read or write memory. A
pointer to an array object is in bounds if it points to an element of
the array object. An array index is in bounds if the index is greater
than or equal to zero and less than the size of the array.

Between 2010 and 2015, buffer overflows accounted for between 10-16\% of
publicly reported security vulnerabilities in the U.S. National
Vulnerability Database each year \cite{NIST2015}. The vulnerabilities have affected
software implemented in C and C++ that is widely used, including the
Windows and Linux operating systems, the Internet Explorer, Chrome, and
Safari web browsers, the Apache web server, the OpenSSL security
library, scripting language implementations for Bash, Ruby, and PHP, and
media playback software such as QuickTime.

Because pointers and array indices are not bounds checked in C, a
programming error involving them may corrupt memory locations used by
the program. The memory locations may hold data that is important to the
computations being done by the program or data that is essential to the
control-flow of the program, such as return address locations and
function pointers. Memory corruption can lead to a program producing
incorrect results or, in the hands of a malicious adversary, the
complete malfunctioning of the program and the takeover of a running
process by the adversary.

This design note describes Checked C, an extension to C that provides
bounds checking for pointers and arrays. There are two obstacles to
adding bounds checking to C. First, it is not clear where to put the
bounds information at runtime. Second, it is not clear how to make the
bounds checking efficient for programs where performance matters. The
solution of changing the representation of all C pointer types and
arrays to carry bounds information is not sufficient in practice. C may
be used at the base of systems where hardware or standards dictate data
layout and data layout cannot be changed. C programs must also
interoperate with existing operating systems and software that require
specific data layouts.

Checked C addresses the bounds checking problem for C by:

\begin{itemize}
\item
  Introducing different pointer types to capture the different ways in
  which pointers are used. The unsafe C pointer type \texttt{*} is kept
  and three new pointer types are added: one for pointers that are never
  used in pointer arithmetic (\ptr), one for \emph{array pointer
  types} (\arrayptr) that are involved in pointer arithmetic,
  and one for pointer types that carry their dynamic bounds with them
  (\arrayview ).
\item
  For array pointer types (\arrayptr), bounds checking is
  placed under programmer control. This differs from languages like
  Java, where bounds checking is completely automatic. A programmer
  declares \emph{bounds}, where the bounds for an \arrayptr\
  variable are given by non-modifying C expressions. These are a subset
  of C expressions that do not modify variables or memory. They include
  local variables, parameters, constant expressions, casts, and
  operators such as addition, subtraction, and address-of (\texttt{\&})
  operators. Static checking ensures that programs declare and maintain
  bounds information properly. The bounds are used at runtime to enforce
  bounds safety, if necessary.
\item
  Introducing different array types to distinguish between arrays whose
  accesses are bounds-checked and existing C arrays whose accesses are
  not bounds-checked. A programmer places the modifier \keyword{checked}
  before the declaration of the bound(s) of the array: \texttt{int x
  checked[5][5]} declares a 2-dimensional array for which all
  accesses will be bounds-checked.
\item
  For structure types with \arrayptr -typed members, a
  programmer declares \emph{member bounds} for those members. A member
  bounds declares the bounds for a member in terms of members in the
  structure type. Member bounds can be suspended temporarily for
  specific variables and objects. Static checking ensures that updates
  to members maintain the member bounds of the members.
\item
  Introducing bounds-safe interfaces to address the problem of
  interoperation between safe code and unsafe code. A bounds-safe
  interface describes the safe interface to unsafe code by declaring
  bounds for unsafe pointers in function signatures and data structures.
  Because it describes a boundary, it has to be ``safe'' and ``unsafe''
  at the same time, depending on what kind of code is using it. The
  interface is trusted in safe code (code that uses safe pointer types).
  Proper usage is enforced via checking at compile time and runtime. For
  unsafe code, the interface is merely descriptive and not enforced by
  language checking. This provides a way to upgrade existing code to
  provide a safe interface without breaking existing users of the code.
\item
  Introducing checked program scopes, where bounds checking is the
  default behavior. In a checked program scope, definitions of variables
  and functions can use safe pointer types and cannot use unsafe pointer
  types. Declarations involving unsafe pointer types must provide
  bounds-safe interfaces. Checked program scopes avoid problems with
  subtle misuse of bounds-safe interfaces.
\item
  Reasoning about the correctness of programs with declared bounds
  sometimes requires reasoning about simple aspects of program behavior.
  To support this, lightweight invariants are added to C. A lightweight
  invariant declares a relation between a variable and a simple
  expression using a relational operator. An example would be the
  statement \texttt{x < y + 5}. Lightweight invariants can be
  declared at variable declarations, at assignment statements, for
  parameters, and for return values. C is extended with rules for
  checking these lightweight invariants. Just as type checking rules are
  defined by the programming language, so are rules for checking
  lightweight invariants. The checking of the correctness of
  programmer-declared bounds is integrated with the checking of
  invariants.
\item
  For the cases where static checking reaches it limits, a programmer
  can introduce dynamic checks that are runtime errors if they fail.
  Dynamic checks use the syntax \keyword{check} \var{e}, where \var{e} is an
  integer-valued expression. A check is similar to an assert, except
  that it is never removed from the program (unless the compiler proves
  it is redundant). It cannot be removed because the integrity of the
  program depends upon it.
\end{itemize}

For an existing C program to be correct, there has to be an
understanding on the part of the programmer as to why pointers and array
indices stay in range. The goals of the design are to let the programmer
write this knowledge down, to formalize existing practices, such as
array pointer parameters being paired with length parameters, and to
check this information.

To simplify bounds and reasoning about bounds, for \arrayview\
and \arrayptr\ types, pointer arithmetic overflow for these
types is considered a runtime error. Pointer arithmetic involving a null
pointer for these types is also a runtime error.

Efficiency is addressed by extending the static checking so that it can
guarantee that specific bounds checks will always succeed at runtime for
\arrayptr\ and \arrayview\ types. The static checking
supports the scenario of simple control-flow enclosing the bounds check
guaranteeing that the bounds check will succeed. For example, a for-loop
may iterate only over values within the declared bounds for an
\arrayptr\ pointer variable.

A problem with incorporating static checking into a programming language
is that static checking needs to be something that compilers can do
quickly and deterministically. Static checking can become very expensive
to do, depending on the language of invariants and the inference that
the compiler is expected to do. For example, Presburger arithmetic is
integer arithmetic restricted only to addition and less than or equal
operations. It is NP-complete to determine whether a formula in the
first-order logic for quantifier-free Presburger arithmetic is
satisfiable (true or false). Even statically checking properties of
simple fragments of real programs can be computationally intractable.

We address this problem two ways. First, the language of invariants is
restricted. The language of invariants does not allow disjunctive
invariants (invariants involving \verb+||+. For example, one
cannot write an invariant that says that variable x has a pointer to a 5
or 7 element array, depending on the value of some Boolean variable b.
The combination of disjunction with conjunction leads to an exponential
size increase when invariants are put into a normal form to check that
two invariants are the same, so it is avoided.

Second, the inference that compilers are expected to do is limited also.
A compiler acts as a \emph{checker} for invariants. It checks that
declared invariants follow from other \emph{declared} invariants, the
program control-flow, intervening assignments, and simple axioms about
invariants, such as transitivity of relational operators. The compiler
does not try to devise invariants or prove the correctness of
invariants; it applies simple local reasoning to check them. The
programmer has to call out the relevant facts. If a programmer declares
an invariant \texttt{x == y} but neglects to declare the invariant
\texttt{y == z}, the checker may not be able to reason several
statements later that \texttt{x == z}, even though it may be true at
that point in the program. This is taking advantage of the fact that
checking a proof is usually much easier than creating the proof.

One question for the design is how usable it is with these restrictions.
We plan to evaluate the limitation on the language of invariants
empirically by examining code and looking for code that requires
disjunctive invariants about bounds. We will also see how many
invariants need to be explicitly declared for the compiler.

Another problem with static checking is that static checking has
theoretical and practical limits. A program logic may be incomplete: a
fact that is true about a program may not be provable in the logic. Even
when a program logic can prove a fact, the proof may require deep
knowledge of an area or the properties of the program. A programmer may
not have the time or expertise to construct the proof that the fact is
true. For example, code may be relying on properties of modular
arithmetic for correctness. For these situations, programs can
dynamically check facts that are believed to be true, but not provable
via the static checking in the language.

Establishing the safety of pointer operations is just the first step in
establishing type safety for C programs. There are other ways which C
programs may fail. C programs may incorrectly deallocate memory that is
still in use, do incorrect unsafe pointer casts, or have concurrency
races that tear data structures. Addressing these problems is beyond the
scope of this design document. For now, we assume that programs are
correct in these other aspects.

The rest of this document is organized as follows. Chapter~\ref{chapter:principles}
describes the principles that will be applied to extending C.  
Chapter~\ref{chapter:core-extensions} describes
the new pointer types and the new array types, including syntax,
semantics, and error conditions. It also covers other extensions to C
semantics. One extension is the introduction of safe program scopes to
prevent inadvertent use of unsafe types. Another extension is a
refinement of C's notion of undefined behavior. We wish to maintain
bounds safety even in the face of integer overflow, which is a
widespread possibility in C programs. This requires a more nuanced
requirement on the result of overflow than ``the result is undefined.''

Chapters~\ref{chapter:tracking-bounds} through \ref{chapter:aliased-members}
present the extensions to C for declaring bounds
for \arrayptr\ values stored in variables and structures and
checking the consistency of those declared bounds. The extensions are
organized by C language feature, moving from simple language features to
more complicated language features. This allows a reader to understand
concepts one at time.

Chapter~\ref{chapter:tracking-bounds} describes how programmers declare the bounds for
\arrayptr\ variables and the meaning of bounds declarations.
Bounds can be declared at variable declarations. They can also be
declared for automatic (local) variables at assignments. If the only
bounds declaration for a variable is at the declaration of the variable,
the bounds declaration is an invariant bounds declaration. An invariant
bounds declaration must be true for the lifetime of the variable. If
there are bounds declarations for the variable at assignments or other
declarations, all bounds declarations for the variable are
dataflow-sensitive. Dataflow-sensitive bounds declarations extend via
dataflow to uses of the variable.

Invariant bounds declarations must usually be valid after every
statement and declaration in the scope of a variable. Sometime multiple
statements are needed to update the set of a variables used in an
invariant bounds declaration. To support this, expression statements and
declarations can be \emph{bundled}, in which case bounds declarations
must be valid only at the end of the bundle. Within the bundled block,
the variables may be inconsistent with respect to bounds declarations
that use them. The variable can be updated so that consistency is
re-established at the end of the bundled block.

Chapter~\ref{chapter:checking-bounds} describes rules that the compiler uses to check the validity
of bounds declarations. It covers inferring bounds for expressions.
Because expression may have assignments embedded within them, it also
covers inferring effects of an expression on the bounds of variables.
This inferred information is then used to validate that declarations and
statements correctly declare bounds and maintain the bounds information.

Chapter~\ref{chapter:interoperation} covers interoperation between 
safe and unsafe code. It covers
conversions between safe and unsafe pointers, as well as the new kinds
of pointers. It pins down the notion of safe code and unsafe code.
Finally, it covers bounds-safe interfaces in depth.

Chapter~\ref{chapter:structure-bounds} extends these ideas from variables to data 
structures by
introducing member bounds, which are type-level invariants about members
of structure types. For now, it assumes that the programmer has done
concurrency control around shared data properly and ignores the fact
that data structures may be modified in racy ways such that invariants
are violated.  Chapter~\ref{chapter:checking-structure-bounds} describes
the rules that the compiler uses to check programs with member bounds.

Chapter~\ref{chapter:aliased-members} extends these ideas to pointers 
to structures.

Chapter~\ref{chapter:simple-invariants}
extends the checking of bounds to incorporate simple
reasoning about bounds and program behavior. It includes a set of rules
for deducing facts that are true about a program at a specific point in
the program (for example, given an assignment \texttt{x = y;} the fact
that x == y is true after the assignment). Facts can also be deduced
from program control-flow. There are additional rules for reasoning
about whether one fact is true given a set of other facts (for example,
given x == y and the statement \texttt{z = y;} z == x is true after that
statement). These rules and facts can be used to deduce the correctness
of bounds declarations that differ from those inferred directly by the
checking described in Chapters~\ref{chapter:checking-bounds} 
and~\ref{chapter:checking-structure-bounds}. For example, a programmer
may wish to narrow the memory that is accessible via an
\arrayptr\ variable by declaring bounds that are a subrange of
the bounds inferred by the checking. A programmer may wish to update the
bounds for an \arrayptr\ variable after an assignment \texttt{x
= y}, substituting x for y.

The same static checking that is used for bounds can be used to reason
about the ranges of variables at specific points in a program. From
there, it is a short step to deducing at compile-time that bounds are
always satisfied at a particular memory access in a program. For
example, a fact can be that the range of an integer variable \texttt{i}
is always between 0 and 10. This can be used to deduce that an array
access in a crucial inner loop is always in bounds.

Chapter~\ref{chapter:pointers-to-pointers}
addresses pointers to \arrayptr\ variables and
address-taken variables.

Chapter~\ref{chapter:lessons} describes prior work on extending C for safety and some
lessons that we drew from that work. We are heavily influenced by the
CCured and Deputy systems.

Chapter~\ref{chapter:eval} evaluates the design by describing our experience modifying
an existing C open-source code base by hand to use the Checked C
extensions. We chose to modify OpenSSL, an important widely-used
open-source code base. The static and dynamic checking is not
implemented in a compiler yet, so we cannot be sure of the correctness
of the modifications or understand the practical benefits of checking.
Still, this gives some idea about the usefulness and applicability of
the design.

Chapter~\ref{chapter:open-issues} summarizes the open problems uncovered by 
Chapter~\ref{chapter:eval} or
unaddressed by the design. It also describes next steps for implementing
this in a C compiler.

Chapter~\ref{chapter:design-alternatives} discusses design alternatives
that were considered and not chosen.  It explains why those alternatives were
not chosen.
