% !Tex root = checkedc.tex

\chapter{Lessons from Existing Research}
\label{chapter:lessons}

There has been substantial research on showing the correctness or type
safety of C programs with pointer arithmetic. The research is a
combination of applied type theory, program logics, program
verification, and dynamic techniques that can be used when static
checking becomes too complicated. The research has not been productized
by industry, even though flaws in C and C++ programs have caused
widespread, costly security problems.  What
lessons can we learn from existing research?

CCured \cite{Necula2005} uses whole-program static analysis to identify 
different uses of pointers in C programs. It identifies pointers that are used to
read or write values only (safe pointers), pointers that are used in pointer
arithmetic also (sequence pointers), and pointers that are involved in
possibly non-type safe casts (wild pointers). It uses a multi-machine
word representation for sequence pointers and wild pointers. It also
changes the representation of data pointed to by wild pointers. This
changes data layouts and causes interoperation problems. The CCured
results clear show that many pointers are never used with pointer
arithmetic. It suggests that new types of pointers should be introduced
into C for pointers that have different safety requirements. This would
be an easy and low-risk way to improve the safety of C programs.

Deputy \cite{Condit2007,Feng2006} uses dependent types to avoid runtime layout changes
in pointers involved in pointer arithmetic. The dependent types allow the bounds of
the pointers to be specified as part of the types of the pointers and
the bounds to depend on runtime values. Deputy requires programmers to
annotate function parameters, data structures, and global parameters
with dependent types. It then infers dependent type annotations for
local variables and adds runtime checks to make the dependently-typed
program type check. The runtime checks enforce that pointer values stay
in bounds. The checks apply to pointer arithmetic and pointer
dereferencing.

Havoc \cite{Condit2009} goes beyond Deputy and allows types to be combined with program
verification. It allows a programmer to specify program invariants that
imply type safety and can verify these invariants statically. It can
handle unsafe code such as using a pointer to a field to access a prior
field in a data structure.

There are some high-level conclusions that we can draw from these
systems. First, it is not feasible in practice to require widespread
data layout changes, as CCured did. Second, this means that use of
program invariants and/or runtime system support will be required to
establish that pointer uses are in bounds. Third, it can be arbitrarily
hard to show the type safety statically of low-level systems
code. The code may be type-safe at runtime. A programmer may intuitively know
that it is type-safe and be able to argue informally for correctness.
However, writing down the invariants may be hard and may require deep
knowledge of program verification techniques. It may also lead to
invariants that are longer and more complicated than the original code.
It may be better for a programmer to just rewrite the code to be type
safe.

Extensions for checked pointer operations will require a trade-off between
dynamic techniques for checking pointer bounds (which are easy to reason
about and understand, but require data layout changes) and static
techniques (which may not be easy to reason about and understand, but do
not require data layout changes).

We can draw the following chart of principles versus techniques:

\begin{longtable}[c]{@{}lll@{}}
\toprule
& Dynamic techniques & Static techniques\tabularnewline
\midrule
\endhead
Minimal & \texttt{Yes} & \texttt{No}\tabularnewline
Succinct and clear & \texttt{Yes} & \texttt{Sometimes}\tabularnewline
Incremental & \texttt{No} & \texttt{Yes}\tabularnewline
\bottomrule
\end{longtable}

One concern about the work is that the more applied work, such as
CCured, Deputy, and SAL, has not been adopted by programmers
enthusiastically. SAL is widely used within Microsoft because it is a
mandated part of the software development process, not because
programmers are enthusiastic about it. It has not been adopted widely
outside of Microsoft. It is unclear why these approaches have not been
adopted. There are many plausible possible reasons, including inertia,
lack of a product-quality toolchain, lack of perceived benefit relative
to the effort, and perhaps being difficult to use. It is important to
aim for something that programmers like. When we are facing design
choices, we recommend using user studies of programmers to evaluate the
choices, instead of relying on hunches or opinions.

\section{Using dependent types to enforce checking of pointer bounds}

Deputy adds dependent types to C to specify and track pointer bounds. A
dependent type is a type that may depend on a value at runtime.
Dependent types are built using type constructors that are applied to
types and values. The type constructors capture specific properties of
runtime values. For example, Deputy introduces a type constructor
\lstinline|Array| that can be applied to an integer value (the length of the
array) and an element type. \lstinline|Array 5 int| describes the type of
integer arrays with 5 elements. The \lstinline|Array| type constructor can be
applied to a program variable or an expression, so the type can depend
on a runtime value. For example, \lstinline|Array n int| describes the type
of integer arrays with n elements.

The following example illustrates the use of dependent types in Deputy.
It is adapted from Figure 3 of the Deputy OSDI paper , which in turn was
adapted from the Deputy version of the Linux e1000 network card driver.

The example has an annotation on the \code{buffer_info} field of
\code{e1000_tx_ring}. The annotation means that
\code{buffer_info} points at an array element and that at least
\code{info_count} elements are valid (specifically, it is valid to
access memory between \code{buffer_info} and \code{buffer_info +
info_count - 1}). This information is part of the type of
\code{buffer_info}.

The type checking rule for adding a pointer of \lstinline|Array| type with an
integer requires that the integer value be in bounds for the array. To
type check \code{tx_ring->buffer_info[i]}, the type
checker must show at compile time that \code{i >= 0} and
\code{i <= tx_ring->count}. In general, taking
an arbitrary program and establishing equality between runtime values at
compile time is undecidable, which means that type checking would be
undecidable. To make type checking decidable, Deputy examines statements
whose type checking rules require establishing facts at compile-time
that involve runtime values. Deputy systematically inserts run time
checks that establish the relationships are true before the statements.
If a relationship between runtime values is not true, the checks cause
the program to fail at run time. Of course, these runtime checks can be
eliminated by later optimization. In this specific case, the compiler
injects a runtime check \lstinline|assert(0 <= i && i < tx_ring->info_count);| into the program. 
In the example, the runtime checks are italicized assert statements.

\begin{lstlisting}
struct e1000_tx_ring {
    ...
    unsigned int info_count;
    struct e1000_buffer * count(info_count)
        buffer_info;
    ...
};

static boolean t
e1000_clean_tx_irq(
    struct e1000_adapter *adapter,
    struct e1000_tx_ring *tx_ring)
{
    ...
    \textit{assert(tx_ring != NULL);}
    spin lock(&tx_ring->tx_lock);
    ...
    i = tx_ring->next_to_clean;
    \textit{assert(0 <= i && i < tx_ring->info_count);}
    eop = tx_ring->buffer_info[i]
    .next to watch;
    ...

    spin unlock(&tx_ring->tx_lock);
}
\end{lstlisting}

\section{Problems with making {C} function bodies dependently-typed}
\label{section:dependent-typing-issues}

C function bodies need to be rewritten to use dependent types and to be
properly typed according to dependent typing rules. Local pointer
variables must be modified to be dependently-typed. As explained
earlier, this means that runtime checks must be added to the code.
Dependently-typing local pointer variables also requires the addition of
computations to track array bounds and local variables to hold the
results of those computations.

Deputy proposes the use of an inference and rewriting step for C
function bodies automatically during compilation to make C function
bodies dependently-typed. Deputy takes a C program, does inference, and
rewrites the programs so that local variables are properly
dependently-typed. Deputy is able to do this automatically given
programs where method parameters, data structures, and global variables
are annotated with dependent types. Deputy introduces local variables
and computations to track missing dependent information. It also
introduces runtime assertions that check that pointers remain within
range. It is an impressive technical feat that this can be done
automatically.

The rewriting step, however, breaks the principle of preserving the
efficiency and control of C. It introduces computations into the program
that may have significant cost and that are not under programmer
control. For C, it is highly desirable that programmers control the
computations that are added to programs to enforce bounds checking. The
rewriting step also potentially violates the principle of clarity
because understanding program failures may require understanding these
computations.

The following code example from Figure 1 of the Deputy ESOP paper shows
this. Here is the original code:
\begin{lstlisting}
int sum (int * buf, int * end) {
    int sum = 0;
    while (buf < end) {
        sum += * buf;
        buf = buf + 1;
    }
    return sum;
}
\end{lstlisting}

Here is the Deputy-generated code, assuming that the function parameters
are annotated with dependent types. The new code inserted by Deputy has
been italicized:

\begin{lstlisting}[escapechar=\@]
int sum (int * count(end - buf) buf, int * end) {
    int sum = 0;
    while (buf < end) \{
        @\textit{assert(0 < end - buf);}@
        sum += * buf;
        int tmplen = (end - buf) - 1;
        @\textit{assert(0 <= 1 <= end - buf);}@
        int * count(tmplen) tmp = buf + 1;
        @\textit{assert(0 <= end - tmp <= tmplen);@ // how to explain this failure??}
        buf = tmp;
    }
    return sum;
}
\end{lstlisting}

The program now has computations of \code{end - buf} and \code{end
- tmp} that track the bounds of the array \code{buf}. A C programmer
would not expect instructions for these computations to be injected into
the program. This makes the program less efficient and these
instructions are also not introduced under programmer control. In
addition it might be difficult for a programmer to understand a bounds
failure in this program at runtime when the failure involves a
synthesized expression such as \lstinline|end - tmp <= tmplen| that did not
exist in the original source code.

This suggests that the automatic inference and rewriting during
compilation that Deputy does is not appropriate for C. We will not do
this for Checked C.

\section{Problems with dependently-typed {C}}

We have ruled out using an automatic inference and rewriting step during
compilation. If we want to use dependent types to enforce bounds
checking in C, this means that programmers will have to write code
directly in a variant of C that is dependently-typed. It would be
possible to use a Deputy-like to automatically rewrite existing C
programs into this dependently-typed C, provided that dependent-type
annotations have been provided for parameters, global variables, and
data structures. Code could be converted and then programmers could
continue developing their software in a dependently-typed C.

There are several problems with using a dependently-typed variant of C
directly for programming. First, dependent types are a big change to the
C type system and C type checking. Dependent types are a rather abstract
concept that may be hard for many programmers to understand. Even if
programmers can understand dependent types, type checking now becomes a
complicated exercise: to type check a dependently-typed statement, the
type checker must prove that certain runtime invariants are true before
the statement. To illustrate this, consider the type checking rule for
variable assignment from \cite{Condit2007}. This
rule requires that the type checker prove that certain invariants must
be true at runtime before the statement. Type checking becomes entangled
in general reasoning about program invariants.

Second, dependent types do not interact nicely with imperative
programming. The program always has to be well-typed according to
dependent-typing rules. To address this, Deputy introduced a programming
operator that may seem odd to a C programmer: a new parallel assignment
operator.

Consider a structure with a pointer field whose size depends on another
field:
\begin{lstlisting}
struct S {
   int * count(len) arr;
   int len;
}
\end{lstlisting}

If there is a local variable of type S, neither field of the variable
can be updated independently of the other field:

\begin{lstlisting}
S y;
int * count(5) tmp = malloc(sizeof(int) 5);
y.arr = tmp;
y.len = 5;
\end{lstlisting}

The assignment to y.arr will fail to type check because y.len is
out-of-date. Reversing the order of assignments does not help. The
solution in Deputy was to introduce a parallel assignment operator. This
is just a specific instance of the problem of initializing a data
structure that satisfies an invariant. In an imperative language,
programmers expect to be able to initialize a data structure using
several assignments and only assert an invariant at the end of
initialization.

Finally, using dependent types makes programs too verbose: explicit
checks to enforce safety have to be inserted all over the code. To our
knowledge, there are no widely-used languages with array-bounds checking
that requires this level of verbosity. In Java and C\#, the checks are
implicitly done. This was the case in older languages such as FORTRAN
and Pascal, as well.

The conclusion is that dependently-typed C at the source level is
not a practical approach. Given that dependent typing is just one way of 
enforcing program invariants,
it suggests that we should explore other ways to enforce program
invariants necessary for bounds safety for C.

\section{Tracking pointer bounds dynamically}

A requirement that local variable array bounds be tracked by program
invariants could lead to many invariants. The Deputy authors observe
that the Deputy inference algorithm for local variables produces a
result similar to fat pointers: ``Deputy introduces a fat representation
for locals, whereas for data structures, function parameters, and global
parameters, the programmer must specify how to compute the metadata from
data already existing in the program.'' The bounds for a pointer are
tracked explicitly using other local variables. This suggests that fat
pointers could be used as an alternative to program invariants for local
variables, in cases where using program invariants makes code too
verbose and there is not a constraint on data layout.

Suppose there is a \spanptr\ type constructor where
\spanptr\ is a struct that has a pointer field and lower and
upper bounds on the valid range of memory associated with a pointer.
\spanptr\ values can be used where pointer values can be
used, with safety checks added for various operations. Consider the
original example in Section~\ref{section:dependent-typing-issues}. 
Here is the code for the method using
\spanptr.

\begin{lstlisting}
int sum (int * count(end - tmpbuf) tmpbuf, int * end) {
    span<int> buf = new span(tmpbuf, tmpbuf, end);
    int sum = 0;
    while (buf < end) {
        sum += *buf;   // checks that buf is in  bounds before dereferencing it        
        buf = buf + 1; // new struct with updated pointer, same bounds
    }
    return sum;
}
\end{lstlisting}

\section{Summary of lessons}

It is clear from the CCured results that many C pointers are never used
with pointer arithmetic. This suggests strongly that new types of
pointers should be introduced into C for pointers that have different
safety requirements. This would be an easy and low-risk improvement to
the safety of C programs.

For pointers where pointer arithmetic is allowed, a combination of
dynamic and static techniques will be needed to ensure bounds-safe pointer
operations. Dynamic techniques that change pointer representations allow
succinct and clear code, at the expense of changing data layout and
causing problems with interoperation and likely efficiency. Static
annotations allow programmers to express bounds information for existing
data structures and parameters and support interoperation. The use of
static annotations for local variables is not required: automatic
inference and insertion of computations to maintain bounds information
and check bounds can be done instead. However, the inference and
automatic insertion of computation compromises the control, efficiency,
and clarity desired by C programmers. This suggests that this approach
should not be used. A more suitable approach for C seems to be to embed
the bounds information for local variables directly into programs.

Static annotations that introduce the complex notion of dependent types
into C programs are problematic. We should explore other formulations of
programs invariants may fit better with C.