% !Tex root = checkedc.tex

\chapter{Type system extensions for improving type safety}

\section{Introduction}

C programs are well-known for taking a loose approach to type safety that
can lead to memory corruption.   There are three C language features that
allow this: pointers to void, casts between different pointer types or
pointer types and integers, and union types.  In this chapter, 
we describe language extensions for replacing  most uses of \uncheckedptrvoid{}
with type-checked code that cannot cause type confusion.

Pointers to void are cast implicitly to and from other pointer types with no checking.
This can lead to type confusion problems, where pointers to objects of one type are mistakenly
assumed to be pointers to objects of different types.  Storing through the pointers
can corrupt memory directly and indirectly.

Pointers to void are used widely for {\em untyped} programming that bypasses
static checking.  Uses include:
\begin{itemize}
\item As handles that hide details of API implementations.  An
API may provide a handle that is typed as \uncheckedptrvoid\ for 
users the API, with the API implementation using a pointer to an actual
type.
\item As pointers to arrays of bytes.  This is done by APIs such as \lstinline+memcpy+.
\item In generic data structures or functions. For example, 
a program may have a data structure and functions for operating on lists. 
The list may  have elements typed as \uncheckedptrvoid\ to allow the code
to be used for many different types of list elements, 
even though in practice only one type of data may be stored in a list instance. 
\item For registering callback functions that are to be provided with 
user-supplied data at the callback.  Callback functions can take \uncheckedptrvoid{} arguments and 
the user-suppled data can be cast to \uncheckedptrvoid{}.
\item To provide a union of pointer types, without changing data representation
\end{itemize}

These extensions are restricted so that code remains data-layout compatible and binary-code
compatible with existing code.  The extensions include:
\begin{itemize}
\item Opaque types: these allow different kinds of handles to be distinguished.
They are incomplete types with enough information to represent
the types, but nothing else.  They can be copied around and used, but the internal
details remain unknown in the scope of the declaration.  No conversions to and 
from opaque types are allowed.

\item Generic structs, functions, and type definitions: these allow structs
or functions to be re-used for many types.   For example,
a generic structure for a  \lstinline+List+ can be re-used for many different types of list
elements.  Generic functions can implement generic list operations such as \lstinline+Append+.

\item Hidden types: these generalize opaque types to handle callbacks involving
user-supplied data.  Hidden types allow programmers to package up a callback function and data and
say that they use  some type \lstinline+T+ whose details are hidden from the code that does
the callback.   Enough details about \lstinline+T+ are available that the callback can be 
implemented in a type-safe fashion efficently.  Hidden types hide details of the type 
of user-supplied data, instead of erasing it like \uncheckedptrvoid{} does.
\end{itemize}

Generic structs and functions can be combined with bounds-checking 
to provide type-safe interfaces for functions that operate on arrays of bytes.
Informally, bounds checking provides a way to guarantee that entire objects are handled,
avoiding corruptions caused by operating only on partial objects.

The remaining sections of this chapter explain the extensions for 
opaque types, generic functions, generic structures, and hidden types.  They also
provide examples that illustrate how the extensions can be used to
replace uses of \uncheckedptrvoid{}.

\section{Opaque types}

We extend \keyword{typedef} declarations so that they can also declare opaque types. 
The \keyword{typedef} is followed by the new keyword \keyword{opaque}.
This creates a new named type that is distinct from all other types during
typechecking and that uses the specified type as its runtime representation.
Importantly, the new named type is not the same as the specified type.  This allows
handle types to be distinguished during typechecking:
\begin{lstlisting}
typedef opaque void *ProcessHandle;
typedef opaque void *FileHandle;
\end{lstlisting}
An attempt to use a \lstinline+ProcessHandle+ where a \lstinline+FileHandle+ 
is expected will result in a typechecking error.

To implement operations on an opaque type,
programmers reveal the actual implementation type of the opaque type
in some scopes of the program.  The
keyword \keyword{typedef} is followed by the new keyword \keyword{reveal}:
\begin{lstlisting}
typedef reveal _Ptr<struct ProcessData> ProcessHandle;
typedef reveal _Ptr<struct FileData> FileHandle;
\end{lstlisting}
In the scope of a \keyword{reveal} declaration, casts between the
opaque type and the implementation type are allowed.  
Converting the implementation type to the representation
type and back to the implementation type must not lose information.  
The rules for conversions in the C specification are followed.   Note that 
this allows the representation type to be larger (in terms of bits) than the implementation type.

C-style casts between opaque types and other types are otherwise not allowed.
In the special case where a programmer must convert an opaque type to another
type (for example, a \uncheckedptrvoid{} is being stored as an integer in an existing
program), the special operator \lstinline|opaque_cast| is provided.

It is implementation-defined whether a pointer to a function type with a
parameter or return type that is an implementation type can be converted
explicitly to a pointer to a function type where the corresponding type
is the opaque type.   The calling convention must be the same for
the implementation and representation type.  If the representation type is \uncheckedptrvoid{}
and the implementation type is also a pointer type,  these conversions would be
allowed in C implementations where all pointer types have the same calling convention.

\section{Generic functions}
\label{sec:generic-functions}

When C programmers use a \uncheckedptrvoid{} pointer so that a function
can be re-used, they typically use the \uncheckedptrvoid{} in place of a pointer 
to some type \var{T}.  At calls to the function, pointers to \var{T} are cast
implicitly to \uncheckedptrvoid{} or back from \uncheckedptrvoid{}.  
Sometimes the function is re-used by casting integers to \uncheckedptrvoid{} pointers.
We want to able to check these uses and ensure they do not cause type confusion.

We want there to be a single copy of a generic function so that programmers
retain low-level control and updated code is binary ompatible with existing code.  
This means that the code cloning as is done in C++ templates cannot be used.  
We instead require that a uniform representation for generic data be used.
Generic data must either be treated as an incomplete type (with no assumption
about size) or it must fit in the space provided by the programmer when
a generic function is defined. There are implementation-dependent restrictions 
on converting generic function pointers to  other function pointer types,
based on calling conventions.

We begin with uses of \uncheckedptrvoid{} to represent \lstinline+T *+ pointers.
The mechanism is simple: we introduce type variables to represent unknown types.
These type variables will be treated as incomplete C types with an unknown
representation (we provide a way to describe representations in a later
section).   A generic function is constructed by parameterizing a
regular function over type variables.  The generic function is applied
to types to create a concrete (non-generic) function that can be called.
A programmer may need to supply information at runtime about the size of T,
so that pointer arithmetic may be done.

We illustrate these ideas by example, starting with \lstinline+bsearch+, 
a C standard library function for binary
searching an array of elements of some type \var{T}.  It
takes a key that is a pointer to T, the array of elements, the number of elements of
the array, the size of T, and a comparison function.  We provide the original declaration
of bsearch and the version modified to use a generic type.  In the modified version,
\lstinline+for_any(T)+ make \lstinline+bsearch+ a generic function.  This means that
\lstinline+bsearch+ works for any type \lstinline+T+.
\begin{lstlisting}
// Original version
void *bsearch(const void *key, const void *base, size_t nmemb, 
              size_t size, int ((*compar)(const void *, const void *)));

// Generic version (not correct)
for_any(T) T *bsearch(const T *key, const T *base, size_t nmemb,
                      size_t size, int ((*compar)(const T *, const T *)));
\end{lstlisting}
To use the generic version of \lstinline+bsearch+, the programmer applies \lstinline+bsearch+ to
a specific type.  For programmer familiarity, we use the C++ syntax for template
instantiations.
\begin{lstlisting}
int arr[] = { 0, 1, 2, 3, 5}
int k = 3;
int cmp(int *a, int *b);
bsearch<int>(&k, arr, sizeof(arr), sizeof(int), cmp);
\end{lstlisting}

It is possible to misuse the generic version of \lstinline+bsearch+ and cause
an out-of-bounds memory access: \lstinline+base+ might not point to a
large enough array or \lstinline+compar+ might treat its arguments as pointers to arrays.
This can be addressed by adding bounds checking.  Bounds checking brings
an interesting problem to light: \lstinline+size+  must match the size of \lstinline+T+.

One might think the solution is that \lstinline+bsearch+ should not take 
\lstinline+size+ as an argument.  The implementation of bsearch could just 
use \lstinline+sizeof(T)+. However, this information is not known at compile-time
within the implementation of \lstinline+bsearch+.   \lstinline+sizeof(T)+
must be passed in as an argument.  To indicate that \lstinline+size+ must
hold \lstinline+sizeof(T)+, we introduce a constraint on the parameter \lstinline+sizes+:
\begin{lstlisting}
// Generic version (correct)
for_any(T) ptr<T> 
    bsearch(ptr<const T> key,
            array_ptr<const T> base : byte_count(nmemb * size),
            size_t nmemb, size_t size : sizeof(T),
            ptr<int (ptr<const T>, ptr<const T>) compar);
\end{lstlisting}
To keep the example simple, we ignore that this function actually needs a 
bounds-safe interface to avoid changing its type for existing code.  
We describe  bounds-safe interfaces for generic
functions in Section~\ref{sec:generic-bounds-safe-interfaces}.

Other functions from the C standard library could be given generic function types too:
\begin{lstlisting}
// Generic versions (not correct)
for_any(T) T *malloc(size_t size);

for_any(T) T *calloc(size_t nmemb, size_t size : sizeof(T))

for_any(T) void free(T *pointer)

for_any(T) T *memcpy(T * restrict dest, const T *src, size_t n);
\end{lstlisting}
These functions need bounds checking also.  A programmer
could pass the wrong value for \lstinline+size+ to \lstinline+malloc+
ore \lstinline+calloc+, for example.  Here are
versions with bounds checking (ignoring for now the need for bounds-safe interfaces):
\begin{lstlisting}
// Generic versions (correct)
for_any(T) array_ptr<T> malloc(size_t size) : byte_count(size);

for_any(T) array_ptr<T> calloc(size_t nmemb, size_t size : 
                               sizeof(T)) : byte_count(nmemb * size);

for_any(T) void free(array_ptr<T> pointer : count(1));

for_any(T) array_ptr<T> memcpy(restrict array_ptr<T> dest : byte_count(n),
                               array_ptr<const T> src, byte_count(n),
                               size_t n where n % sizeof(T) == 0) :
                               byte_count(n);
\end{lstlisting}
In the case of \lstinline+malloc+, if the size is not a multiple of the size of \lstinline+T+,
only enough space for part of an object of type \lstinline+T+ will be allocated.  With the 
bounds-safe interface, though, the program will only be able to read or write the 
partially allocated space.   Providing a type-safe interface to \lstinline+memcpy+
is more challenging.   It is incorrect to copy only part of an element of T.
\lstinline+T+ could be struct that contains a pointer.  Copying only a few bytes
of the pointer could result in an invalid pointer.
This is handled in the bounds-safe interface by requiring that the
size be a multiple of the size of \lstinline+T+.

\section{Type variables with representation constraints}

The code for \lstinline+bsearch+ can be re-used because \lstinline+bsearch+ is
not making assumptions about the runtime representation of data with a type
given by type variable \var{T}.  However, sometimes programmers do make assumptions 
about the representation of \var{T}.  They may write a hash table that maps
keys to pointers or integers.    A programmer may specify the representation type by
adding a {\it representation} clause to the declaration of a type variable.   The
type variable name is followed by \lstinline+: rep(+ \var{type name}\lstinline+)+.
For example, we may have an add function for hash table that maps integers to data:
\begin{lstlisting}
for_any (Data : rep(void *))
   hash_add(struct HashTable<T> *table, int key, Data d);
\end{lstlisting}
The function \lstinline+hash_add+ can only hold data that fits within a 
\uncheckedptrvoid{} pointer.  When a type variable has a representation type specified, 
there must be a lossless conversion from types supplied as type arguments for
a type variable to the representation type and back.
This is the same restriction placed on 
representation and implementation types for opaque types.   In this example, on a
32-bit architecture where \lstinline+int+ is the same size as +\uncheckedptrvoid{}, 
\lstinline+hash_add+ could be applied to \lstinline+int+.  It could not be applied to
to 64-bit integers:
\begin{lstlisting}
hash_add<int>(t, key, 314);            // allowed.
hash_add<long long>(t, key, 314L);     // not allowed.
\end{lstlisting}
The compiler will implicitly insert conversions to the type representation
at calls to the generic function where the type of a parameter is given
by a type variable.  It will insert conversions from the type representation
when the return is given by a type variable.

At uses other than as a call target, it is implementation-defined whether a pointer to a function type 
created by applying a generic function to type arguments can be used where
a pointer to a function type with the monomorphic type (the function type obtained
by substituting the type arguments for the type variables) is expected.  If a type
variable occurs as the type of a parameter or the return, the corresponding type argument
for the type variable must have the same calling convention as the type variable's
representation type.   Non-generic and applied generic functions with the same function type
need to agree on their calling conventions.

As an example, floating-point types may have a different calling convention
than integer or pointer types.  It makes sense for \lstinline+hash_add+ to allow
hash tables that map keys to \keyword{float} values, if \keyword{float}
values fit into a \keyword{void *}.  The compiler can insert  
bitwise-conversions at calls and returns.   However, \lstinline+hash_add<float>+
is not interchangeable with a function \lstinline+f(struct HashTable<T> *table, int key, float f)+.
\lstinline+hash_add+ expects its argument to be passed the same way that \uncheckedptrvoid{} 
is passed.

\section{Generic structures}
We use the example of a generic \lstinline+List+ structure.   In the case of structs,
the \forany{} clause comes after the tag name of the structure.  
Section~\ref{section:foranyalternatives} explains why the \forany{} clauses 
are placed differently.
\begin{lstlisting}
struct List for_any(T) { 
   T *elem;
   List<T> *next;
}
\end{lstlisting}
A function could take a pointer to a list of T and compute its length:
\begin{lstlisting}
for_any(T) len(List<T> *head) {
  int count = 0;
  while (head != null) {
     count++;
     head = head->next;
  }
  return count;
}
\end{lstlisting}
Within the declaration of \lstinline+List+, we do not allow polymorphic recursion.

Note that for a type variable to be used as the type of a struct member, 
the type variable must have a representation type.   Otherwise the type
variable is an incomplete type.  C does not allow members to have incomplete
types.

\section{Grammar changes for generic functions and structs}

The grammar from the C11 specification \cite{ISO2011} is extended to allow
generic functions:
\begin{tabbing}
\var{declarat}\=\var{ion:}\\
\>\var{declaration-specifiers} \var{init-declarator-list} \texttt{;} \\
\>\ldots{} \\
\\
\var{function-definition:}\\
\>\var{declaration-specifiers} \var{declarator}  
  \var{declaration-list}\textsubscript{opt} \var{compound-statement}\\
\\
\var{declaration-specifiers:}\\
\>\var{for-any-specifier} \var{declaration-specifiers}\textsubscript{opt} \\
\>\ldots{} \\
\\
\var{for-any-specifier:}\\
\>\texttt{\_For\_any (} \var{type-variable-list} \texttt{)} \\
\\
\var{type-variable-list:} \\
\>\var{type-variable} \\
\>\var{type-variable} \texttt{,} \var{type-variable-list}\\
\\
\var{type-variable:} \\
\>\var{identifier} \\
\\
\\
\var{type-specifier:} \\
\>\var{type-variable} \\
\>\ldots{} \\
\end{tabbing}
At most one \var{for-any-specifier} may occur in the list of declaration specifiers
for a declaration or function definition.
The \var{for-any-specifier} introduces a list of type variables into scope.  
The type variables are available in any following \var{declaration-specifiers} that are part
of the declaration.  The scope of the type variables extends to the end of
the declaration or function definition.

\section{Hidden types}
This section is to be filled in.

