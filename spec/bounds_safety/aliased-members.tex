% !Tex root = checkedc.tex

\chapter{Member bounds and pointers to structure types}
\label{chapter:aliased-members}

A member bounds may be suspended for an object that is referenced using
a pointer. The syntax for this is:

\var{fact ::=}

\texttt{suspends(}\var{exp}\texttt{-\textgreater{}}\var{identifier}\texttt{)}

\texttt{holds(}\var{exp}\texttt{-\textgreater{}}\var{identifier\texttt{)}}

\var{exp} must have type \ptrinst{\textit{S}}, where
\var{S} is some struct type, and \var{identifier} must be a field of
\var{S}. In practice facts of the form (*\var{exp}).\var{identifier}
will be allowed also. In this description,
(*\var{exp}).\var{identifier} is omitted because it is semantically
identical to \var{exp}\texttt{-\textgreater{}}\var{identifier}.

A member bounds may be suspended for an expression on entry to a
function or on exit to a function. If it is suspended on exit, the
expression must involve only the return value and parameter variables
and the parameter variables cannot be changed during the body of the
method.

Here are examples of several functions with suspended member bounds on
function entry and exit:

\begin{verbatim}
struct S {
     array_ptr<int> a : count(len);
     int len;
}

// Member bounds for arg is suspended and remains suspended
// on exit
void f1(S *arg where suspends(arg->a)) where suspends(arg->a)
{
   ...
}

 // Member bounds for arg is suspended on entry and holds on exit
void *f2(S *arg where suspends(arg->a))
{ 
     ...
}

// Member bounds is suspended for the return value.
S *f3() where suspends(return->a)
{
    ...
}

// Member bounds is suspended for the argument on entry.
// It is suspended on exit for the return value and the argument.
S *f3(S *arg) where suspends(return->a) &&  suspends(arg->a)
{
    ...
}
\end{verbatim}

While a member bounds is suspended for an object, the
\arrayptr\ field of that object covered by that invariant
cannot be used to access memory. The possibility of aliasing makes
enforcing this rule complicated. Other variables or memory locations may
hold pointers to the object pointed to by \var{exp}. The
\texttt{suspends} and \texttt{resume} annotations apply to \var{exp}
only. The other variables or memory locations may not be annotated as
having suspended member bounds, so they could be used to access memory
using an \arrayptr\ field in an inconsistent fashion.

To prevent this, restrictions are placed on what code can do while a
member bounds is suspended. The restrictions make conservative
assumptions about the possible aliases of exp. In practice, member
bounds should only be suspended for brief pieces of code.
Section~\ref{section:avoiding-aliased-accesses}
describes the rules that enforce these restrictions. 
Section~\ref{section:better-alias-information}
contains proposals for loosening these restrictions by relying on
additional information about pointer aliasing.

We assume for now that the address of a field in a structure type that
is used in a member bounds cannot be taken. If it could be taken, it
would be trivial to break the member bounds.

\section{Avoiding aliased accesses to object members with suspended bounds}
\label{section:avoiding-aliased-accesses}

To allow efficient compilation, it is important to require only local
analyses of aliasing behavior and to avoid requiring whole-program
analyses of aliasing behavior. The following conservative assumptions
are made about possible aliasing to allow local analysis:

\begin{itemize}
\item
  Variables of structure type represent distinct objects from each
  other.
\item
  Only variables that are address-taken can be have pointer aliases to
  their storage. A variable is address-taken if the \& (address-of)
  operator is used to obtain its address or the address of a member of
  the variable.
\item
  Any two pointers to structures can be aliases for the same object,
  even if the pointers point to different structure types. It is
  possible to cast from \ptrinst{\textit{S}} to
  \ptrT if \var{S} and
  \var{T} are compatible types. Autobahn Design Note 3 will cover
  pointer casts and rules for compatible types. For now, with no
  definition of compatible types, we will assume that all structure
  types are potentially compatible.
\end{itemize}

The following rules enforce that an \arrayptr\ field of an
object is not used to access memory while the type-level invariant for
the object is suspended.

We will all paths in a function that start at a point of suspension of a
type-level bounds via a pointer. The point of suspension may be a
statement with an annotation
\texttt{suspends}(\var{exp}\textgreater{}\var{identifier}\texttt{)}.
It may also be the entry to the function, if
\texttt{suspends}(\var{exp}-\textgreater{}\var{identifier}) is true on
entry to the function.

\begin{enumerate}
\item
  For every path from the suspend point, either

  \begin{enumerate}
  \item
    There must be a statement with a
    \texttt{holds(}\var{exp}-\textgreater{}\var{identifier}\texttt{)}
    annotation along the path.
  \item
    Or the path must end at an exit point of the function (a
    \texttt{return} statement, or for a \texttt{void} function that
    returns nothing, the end of the function) and the type-level
    invariant for \var{exp} must be suspended on exit to the function.
  \end{enumerate}
\item
  The limited part of the path will be defined as either

  \begin{enumerate}
  \item
    The part of path between the suspend point and the \texttt{holds}
    expression, if there is one.
  \item
    The entire path otherwise.
  \end{enumerate}
\item
  For the limited part of the path,

  \begin{enumerate}
  \item
    There cannot be a memory read or write through an
    \arrayptr\ member accessed via a structure operator
    \texttt{-\textgreater{}}.
  \item
    There cannot be a memory read or write through an
    \arrayptr\ member of an address-taken variable that was in
    scope at the point of suspension.
  \item
    Any function call must either

    \begin{enumerate}
    \item
      Take \texttt{exp} as an argument, to ensure that 3a and 3b are
      followed while the type-level member bounds is suspended for
      \texttt{exp}.
    \item
      Or the called function must be annotated as following rules 3a and
      3b above.
    \end{enumerate}
  \item
    The value of \texttt{exp} cannot change:

    \begin{enumerate}
    \item
      There cannot be an assignment to a variable used in \texttt{exp}
    \item
      exp cannot read memory that is modified by the limited part of the
      path. For now, this will be approximated the rule that
      \texttt{exp} cannot contain a pointer dereference or a function
      call.
    \end{enumerate}
  \end{enumerate}
\end{enumerate}

We also need to extend facts and checking of facts to handle equality of
pointer-dereferenced members and variables:

\var{fact} ::=

\var{exp}\texttt{-\textgreater{}}\var{identifier} == var

\var{var} \texttt{==} exp\texttt{-\textgreater{}}\var{identifier}

The following kinds of assignments invalidate these types of facts:

\begin{itemize}
\item
  Assignment through an unsafe pointer.
\item
  Assignment to any structure member through a pointer expression other
  than \var{exp}. The other expression could be an alias for the object
  pointed to by \var{exp}. This can happen even if the other expression
  involves a different struct type than \var{exp}.
\item
  Any assignment to a member of an address-taken variable. \var{exp}
  could be an alias for the address-taken variable.
\end{itemize}

\section{Examples}
\section{Loosening restrictions with better alias information}
\label{section:better-alias-information}
