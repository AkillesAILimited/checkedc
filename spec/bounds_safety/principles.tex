% !Tex root = checkedc.tex

\chapter{Principles for language extensions}\label{principles-for-language-extensions}

Here are the principles that are applied to extending C:

\begin{enumerate}
\item
  Preserve the efficiency and control of C. C is designed to be
  low-level and work with the same types that computer processors work
  with. This makes C ``close to the hardware'', allowing programmers to
  write efficient programs and control what programs do precisely at the
  machine level. These features are one reason why C is valued as a
  system programming language. Extensions will be ``pay-as-you-go'' and
  continue to provide precise control to programmers at the machine
  level. Hidden costs will be avoided.
\item
  Be Minimal. This means adding the minimal set of extensions needed to
  accomplish the goals. It is easier to learn extensions if there are as
  few of them as possible. It also lets us stay true to the design goals
  of C.
\item
  Aim for clarity and succinctness. Clarity means that code is easy to
  understand and extensions are straightforward to understand.
  Succinctness means the programmers have less to read or type.
  Programmers value clarity and succinctness because it makes them more
  productive at their jobs. Sometimes clarity and succinctness are in
  tension and sometimes they are not. When they are in conflict, clarity
  will be prioritized above succinctness, primarily because source code
  is read many more times than it is written.
\item
  Enable incremental use. Real systems are large and complicated, with
  hundreds of thousands and millions of lines of code. The teams that
  work on those systems will adopt safe pointer operations over time,
  not all at once, so incremental use of safe pointer operations will be
  supported. Teams will prefer incremental conversion paths because of
  practical matters such as reducing risk, fixing existing bugs
  identified by introducing bounds checking, maintaining system
  stability, and understanding performance effects. Even though
  incremental use will be supported, it is not the end goal. We believe
  that benefits of using safe pointer operations will be modest until
  almost an entire system is converted. At that point, we expect a
  qualitative increase in system reliability and programmer
  productivity.
\end{enumerate}

Two specific design principles are adopted based on these principles:

\begin{enumerate}
\item
  Do not change the meaning of existing C code. Methods that do not use
  extensions will continue to compile, link, and run ``as is''. If the
  meaning of existing C code is changed, it will violate the principles
  of clarity and enabling incremental adoption.
\item
  Adopt existing notations from C++ when it meets our needs, instead of
  inventing new notations. Many systems are hybrid C/C++ systems, so
  this approach fits with the principle of clarity. It also enables
  incremental adoption. One of the design goals of C++ has been that C
  is a subset of C++. That will continue to be true even for our
  extended C.
\end{enumerate}
